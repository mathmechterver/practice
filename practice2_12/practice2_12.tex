\documentclass[a4paper, 14pt]{extarticle}

%% Language and font encodings
\usepackage[english, russian]{babel}
\usepackage[utf8]{inputenc}
\usepackage{amsmath}

\usepackage[a4paper,top=1cm,bottom=1cm,left=1cm,right=1cm,margin=15mm, lmargin=30mm]{geometry}
\title{practice2old}
\author{samstikhin}
\date{September 2018}

\begin{document}
\subsection*{Марковские цепи. ЗБЧ для марковских}
\begin{enumerate}
\item Рассмотрим марковскую цепь(автомат) с 2 состояниями и матрицей переходов.

$$Q = \begin{bmatrix}
        1 - a & a \\
        b & 1 - b 
\end{bmatrix} ~~~~(0 < a+b < 1)$$

Попадая в первое состояние мы пишем в строку $a$, во второе - $b$.
Начальное распределение $\mu = \Big(\frac{1}{2}, \frac{1}{2}\Big)$. 
Найти 
\begin{enumerate}
\item $Q^{n}$
\item Cтационарное распределение
\item Отношение количеств букв $a$ к буквам $b$ в бесконечной строке. 
\item Отношение количество пар $ab$ к парам $bb$ в бесконечной строке.
\end{enumerate}

\item (Команда X сдаёт задачу за малый промежуток времени $\Delta t$ с 
вероятностью $p = \lambda \Delta t + o(\Delta t)$ при $\Delta t \to 0$. 
За этот промежуток времени они могут сдать больше одной задачи: 
вероятность этого равна $o(\Delta t)$. 
В начале соревнования у ребят нет решённых задач. 
Пусть $X(t)$ - случайная величина, показывающая, 
сколько задач команда сдаст за время t. Найдите распределение $X(t)$. 
Покажите, что данная мат-модель является марковским процессом с непрерывным временем, 
где число задач - состояния, t - время.

\item Предположим, что временные интервалы между последовательными приходами в магазин 
посетителей — независимые случайные величины, имеющие показательное распределение 
с параметром $\lambda$. Постройте для состояний из $\{0, 1, 2, \ldots \}$ (число посетителей) 
марковскую цепь с непрерывным временем. 
Найдите вероятность того, что к моменту времени 5 придут ровно k посетителей.

\item 

\end{enumerate}

\end{document}
