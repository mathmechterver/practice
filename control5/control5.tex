\documentclass[a4paper, 14pt]{extarticle}

%% Language and font encodings
\usepackage[english, russian]{babel}
\usepackage[utf8]{inputenc}
\usepackage{amsfonts,amssymb,amsmath}

\usepackage[a4paper,top=2cm,bottom=2cm,left=2cm,right=1.5cm,margin=15mm, lmargin=30mm]{geometry}

\begin{document}

\section*{Контрольная 4. Распределения, сходимости.} 
\begin{enumerate}

\item (3)Пусть $\hat F(x)$ --- эмпирическая функция распределения. Пусть $x$, $y \in \mathbb{R}$. Найдите ковариацию $Cov(\hat F(x), \hat F(y))$.

\item (3)Найти оценку максимального правдоподобия параметра $\theta$ у распределения с плотностью

$$f_{\theta}(y) = \theta y^{\theta - 1} ~~~~ y\in[0,1]$$

\item (3)Пусть  $\{X_1 ,\ldots , X_{n}\}$ — выборка из распределения Бернулли
с параметром p. Является ли статистика $\hat{p} = (\overline{X})^2$ несмещённой
оценкой параметра p? Состоятельной?

\item (3)Пусть $\{X_1 ,\ldots , X_{3n}\}$ — выборка объёма 3n из нормального
распределения со средним a и единичной дисперсией. Проверить
несмещённость и состоятельность следующих оценок параметра a
$$\frac{1}{n}\sum_{i=n+1}^{2n}X_i$$
$$\frac{1}{n}\sum_{i=1}^{n}X_{3i}$$

\item (3)Пусть $\{X_1 ,\ldots , X_{n}\}$ — выборка из равномерного распределения на отрезке $[0, \theta]$. Проверить состоятельность и несмещённость
следующих оценок параметра $\theta$
$$(n+1)X_{(1)}$$
$$\frac{n+1}{n}X_{(n)}$$

\item (3)Пусть $\{X_1 ,\ldots , X_{n}\}$ — выборка из равномерного распределения на отрезке $[0, \theta]$. Является ли оценка $(n+1)X_{(1)}$ 
асимптотически нормальной оценкой параметра $\theta$?

\item (3+3)Найти методом моментов и методом максимального правдоподобия оценки параметров распределениея Вейбула

$$f_{X}(x) = \begin{cases}
\frac{k}{\lambda}(\frac{x}{\lambda})^{k-1}e^{-(\frac{x}{\lambda})^{k}} & x\geq 0 \\
0 & x<0 
\end{cases}$$

\item (4)Распределение Кэптейна определяется плотностью
$$f_{\theta}(y) = \frac{g^{'}(y)}{\sqrt{2pi}}e^{-\frac{(\theta - g(y))^2}{2}}$$
где $g(y)$ — неубывающая дифференцируемая функция. 
Найти оценку максимального правдоподобия параметра $\theta$.

\item У Саши есть 2 классификатора: получше и похуже. 
Предположим, что качество одного классификатора = c и не известно. 
Классификатор тем точнее и лучше, чем больше c. Предположим, что при измерении  
качества мы наблюдаем значения $y_i = c + \epsilon_i$. $\epsilon_i$ - 
независимые одинаково распределённые нормальные величины с некоторой 
дисперсией и нулевым математическим ожиданием. Таким образом, $y_i$
 - тоже случайные величины с мат. ожиданием c и дисперсией $\sigma^2$.
\begin{enumerate}
\item (3) Постройте доверительный интервал для значения c так, чтобы вероятность того, что c лежит в этом интервале была не менее 0.95. Найдите границы интервалов для данных каждого классификатора.

\item (3) Любым известным вам способом (хоть с лекций, хоть с практик, хоть из гугла) проверьте гипотезу о том, что классификатор получше работает точнее, чем тот, что похуже. Можно ли утверждать, что лучший классификатор точнее.
\end{enumerate}
Измерения классификатора получше:
0.64175111, 0.63247873, 0.63313111, 0.63270667, 0.63184   ,
0.64238667, 0.63818667, 0.64000529, 0.63401333, 0.63696317,
0.63300127, 0.63815111, 0.63456127, 0.63844444, 0.64431556,
0.64572444, 0.63088381, 0.63283492, 0.64132952, 0.6414    ,
0.631     , 0.63810667, 0.64361651, 0.63152762, 0.64319556,
0.64393778, 0.64226286, 0.63413079, 0.63395556, 0.62351175,
0.63728095, 0.63190349, 0.63716   , 0.63773397, 0.63676381,
0.63008063, 0.63776952, 0.63110952, 0.63800444, 0.63679111

Измерения классификатора похуже:
0.62203002, 0.62244152, 0.62315778, 0.6215295 , 0.62170169,
0.62271204, 0.62261512, 0.62374008, 0.62170984, 0.62257677,
0.61856799, 0.6150659 , 0.61587726, 0.62742243, 0.6164587 ,
0.62721114, 0.62420552, 0.62223767, 0.62851293, 0.61790504,
0.61169198, 0.62410424, 0.62155702, 0.61685259, 0.61949887,
0.62610344, 0.62347114, 0.61816931, 0.62375206, 0.61676952,
0.62067124, 0.6205655 , 0.61914936, 0.62486339, 0.61320572,
0.61730455, 0.62041464, 0.61807048, 0.62044667, 0.61814111

\end{enumerate}
\end{document}

