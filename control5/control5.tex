\documentclass[a4paper, 14pt]{extarticle}

%% Language and font encodings
\usepackage[english, russian]{babel}
\usepackage[utf8]{inputenc}
\usepackage{amsfonts,amssymb,amsmath}

\usepackage[a4paper,top=2cm,bottom=2cm,left=2cm,right=1.5cm,margin=15mm, lmargin=30mm]{geometry}

\begin{document}

\section*{Контрольная 4. Распределения, сходимости.} 
\begin{enumerate}

\item Пусть $\hat F(x)$ --- эмпирическая функция распределения. Пусть $x$, $y \in \mathbb{R}$. Найдите ковариацию $Cov(\hat F(x), \hat F(y))$.

\item Найти оценку максимального правдоподобия параметра $\theta$ у распределения с плотностью

$$f_{\theta}(y) = \theta y^{\theta - 1} ~~~~ y\in[0,1]$$

\item Найти методом моментов и методом максимального правдоподобия оценки параметров распределениея Вейбула

$$f_{X}(x) = \begin{cases}
\frac{k}{\lambda}(\frac{x}{\lambda})^{k-1}e^{-(\frac{x}{\lambda})^{k}} & x\geq 0 \\
0 & x<0 
\end{cases}$$

\item Распределение Кэптейна определяется плотностью
$$f_{\theta}(y) = \frac{g^{'}(y)}{\sqrt{2pi}}e^{-\frac{(\theta - g(y))^2}{2}}$$
где $g(y)$ — неубывающая дифференцируемая функция. 
Найти оценку максимального правдоподобия параметра $\theta$.

\item Пусть  $\{X 1 ,\ldots , X_{n}\}$ — выборка из распределения Бернулли
с параметром p. Является ли статистика $\hat{p} = (\overline{X})^2$ несмещённой
оценкой параметра p? Состоятельной?

\item Пусть $\{X 1 ,\ldots , X_{3n}\}$ — выборка объёма 3n из нормального
распределения со средним a и единичной дисперсией. Проверить
несмещённость и состоятельность следующих оценок параметра a
$$\frac{1}{n}\sum_{i=n+1}^{2n}X_i$$
$$\frac{1}{n}\sum_{i=1}^{n}X_{3i}$$

\item Пусть $\{X 1 ,\ldots , X_{n}\}$ — выборка из равномерного распределе-
ния на отрезке $[0, \theta]$. Проверить состоятельность и несмещённость
следующих оценок параметра $\theta$
$$(n+1)X_{(1)}$$
$$\frac{n+1}{n}X_{(n)}$$

\item Пусть $\{X 1 ,\ldots , X_{n}\}$ — выборка из равномерного распреде-
ления на отрезке $[0, \theta]$. Является ли оценка $(n+1)X_{(1)}$ 
асимптотически нормальной оценкой параметра $\theta$?

\end{enumerate}
\end{document}

