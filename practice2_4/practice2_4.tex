\documentclass[a4paper, 14pt]{extarticle}

%% Language and font encodings
\usepackage[english, russian]{babel}
\usepackage[utf8]{inputenc}

\usepackage[a4paper,top=1cm,bottom=1cm,left=1cm,right=1.5cm,margin=10mm, lmargin=15mm]{geometry}

\usepackage{amsfonts,amssymb,amsmath}
\usepackage{nopageno, comment}
\usepackage{cmap}
\usepackage{ifthen}
\usepackage{indentfirst}
\usepackage{float}
\usepackage{tikz}
\usepackage{wrapfig}

\title{practice2_4}
\author{Sam Stikhin}
\date{February 2019}

\begin{document}

\section*{2.4 ЦПТ. Локальная и интегральная т. Муавра-Лапласа}
\subsection*{Пререквизиты}
\subsubsection*{Локальная т. Муавра-Лапласа} 
Вероятность $m$ успехов в схеме бернулли из n испытаний примерно равна:
$$P_n(m)  \approx \frac{1}{\sqrt{npq}} \phi(x)$$

$$\phi(x) = \frac{1}{\sqrt{2\pi}}e^{-\frac{x^2}{2}}$$
$$x = \frac{m - np}{\sqrt{npq}}$$
чем ближе вероятность к 0.5, тем точнее результат 


\subsubsection*{Интегральная т. Муавра-Лапласа}
$\xi$ - количество успехов в схеме Бернулли. Вероятность от $k$ до $m$ успехов в схеме бернулли из n испытаний примерно равна
$$P_n(k\leq \xi \leq m)  \approx  \Phi\Big(\frac{m-np}{\sqrt{npq}}\Big) - \Phi\Big(\frac{k-np}{\sqrt{npq}}\Big)$$

$$\Phi(x) = \frac{1}{\sqrt{2\pi}}\int_{-\infty}^{x}e^{-\frac{t^2}{2}}dt$$



\subsubsection*{ЦПТ}
Пусть $X_1, \dots , X_n,  \dots ,$  --- независимые одинаково распределённые случайные величины, имеющих конечное математическое ожидание $\mu$ и дисперсию $\sigma^{2}$. Пусть $S_n={X_1+ \dots  +X_n}.$ Тогда $Z_n=\frac{S_n-\mu n}{\sigma\sqrt{n}}$ сходится к $N(0,1)$ по распределению при $n\to\infty.$

(локальная ЦПТ) Если распределение $X_1$ абсолютно непрерывно, то плотность $f_{Z_n}$ существует и $f_{Z_n}(x)\to \phi(x)=\frac{1}{\sqrt{2\pi}}e^{-x^2/2}$ для всех $x.$

(равномерная оценка: Неравенство Берри-Эссеена) Если $E |X^3|=\rho$ конечно, то для всех $n,x$ $$\Big|F_{Z_n}(x)-F_{N(0,1)}(x)\Big|=\bigg|F_{Z_n}(x)-\int_{-\infty}^{x}\frac{1}{\sqrt{2\pi}}e^{-y^2/2}dy\bigg|\leq \frac{0,5 \rho}{\sigma^3 \sqrt{n}}.$$
	

\newpage
\section*{2.4 ЦПТ. Локальная и интегральная т. Муавра-Лапласа}
\subsection*{Практика}
\begin{enumerate}
\item Вероятность рождения мальчика равна 0,51. Найдите вероятность того, что среди 1000 новорожденных окажется ровно 500 мальчиков.
\item В театре 1600 мест и 2 гардероба. Посетитель выбирает гардероб равновероятно. Сколько в них должно быть мест, чтобы их могло не хватить не чаще раз в месяц.
\item В рулетке есть 37 клеток. По 18 красных и черных и одна зеленая. При ставке выбирается либо черная, либо красная клетка и на нее ставится единичная ставка.
В случае победы - ставка удваивается, в случае проигрыша - все теряется.
Найти вероятность сохранить или преумножить свой капитал в рулетке если сделано 200 ставок.
\item На церемонию вручения дипломов в половине случаев приходят оба родителя выпускника, в
трети случаев один из родителей, а с вероятностью 1/6 не придет никто. В новом году будут вы-
пускаться 600 человек. С какой вероятностью можно утверждать, что родителей будет больше, чем
выпускников?
\item Найти оценку неравенством Бэрри-Эссена для задач с рулекткой и дипломами.


\end{enumerate}
\newpage
\section*{2.4 ЦПТ. Локальная и интегральная т. Муавра-Лапласа}
\subsection*{Домашка}
\begin{enumerate}
\item (1)Имеется 1000 параллепипедов, у каждого из которых длинакаждой стороны может принимать значения
$\frac{1}{2}$ и 1 с вероятностями 0,3 и 0,7 соответственно. Пусть V - суммарный объем этих параллепипедов. 
Оценить вероятность того, что $580< V <605$.
\item (1)В стране насчитывается 10 млн. избирателей, из которых 5,5
млн. принадлежит к партии А, и 4,5 млн. принадлежит к партии В.
Назначаются жребием 20000 выборщиков. Какова вероятность того, что
большинство выборщиков окажется сторонниками партии В?
\item (1)Посмотрим еще раз на задачу с рулеткой из практики. При каком количестве ставок вероятность проигрыша будет меньше $\frac{1}{3}$/ меньше $\frac{1}{4}$
\item (1)Стрелок попадает при выстреле по мишени в десятку с вероятностью 0,5; в девятку —
0,3; в восьмерку — 0,1; в семерку— 0,05; в шестерку — 0,05. Стрелок сделал 100 выстрелов. Оцените
вероятность того, что он набрал более 980 очков; более 950 очков?
\item (1)В поселке 2500 жителей. Каждый из них примерно 6 раз в
месяц ездит на поезде в город, выбирая дни поездок по случайным мотивам независимо от остальных. 
Какой наименьшей вместимостью должен обладать поезд, чтобы он переполнялся в среднем не чаще одного
раза в 100 дней (поезд ходит раз в сутки).
\item (2)Найти оценку неравенством Бэрри-Эссена для задачи со стрелком и поездом.
\item (2)Мера длины фут, как видно из названия, — длина ступни. Но, как известно, размеры ног
бывают разные. Немцы в XVI в. выходили из положения следующим способом. В воскрестный день
ставили рядом 16 первых вышедших из церкви мужчин. Сумма длин их левых ступней делилась на
16 . Средняя длина и была «правильным и законным футом». Известно, что размер стопы взрослого
мужчины — случайная величина, имеющая нормальное распределение со средним значением 262, 5
мм и квадратичным отклонением 12 мм. Найти вероятность того, что два “правильных и законных”
значения фута, определенных по двум различным группам мужчин, отличаются более чем на 5 мм.
Сколько нужно было бы взять мужчин для того, чтобы с вероятностью не менее 0, 99 средний размер
их ступней отличался от 262, 5 мм менее чем на 0, 5 мм?

\end{enumerate}

\end{document}

