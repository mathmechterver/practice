\documentclass[a4paper, 14pt]{extarticle}

%% Language and font encodings
\usepackage[english, russian]{babel}
\usepackage[utf8]{inputenc}

\usepackage[a4paper,top=1cm,bottom=1cm,left=1cm,right=1.5cm,margin=10mm, lmargin=15mm]{geometry}

\usepackage{amsfonts,amssymb,amsmath}
\usepackage{nopageno, comment}
\usepackage{cmap}
\usepackage{ifthen}
\usepackage{indentfirst}
\usepackage{float}
\usepackage{tikz}
\usepackage{wrapfig}

\title{practice2_4}
\author{Sam Stikhin}
\date{February 2019}

\begin{document}

\section*{2.4 ЦПТ. Локальная и интегральная т. Муавра-Лапласа}
\subsection*{Пререквизиты}
\textbf{Локальная т. Муавра-Лапласа} Вероятность $m$ успехов в схеме бернулли из n испытаний примерно равна:
$$P_n(m)  \approx \frac{1}{\sqrt{npq}} \phi(x)$$

$$\phi(x) = \frac{1}{\sqrt{2\pi}}e^{-\frac{x^2}{2}}$$
$$x = \frac{m - n}{\sqrt{npq}}$$
чем ближе вероятность к 0.5, тем точнее результат 


\textbf{Интегральная т. Муавра-Лапласа}
$\xi$ - количество успехов в схеме Бернулли. Вероятность от $k$ до $m$ успехов в схеме бернулли из n испытаний примерно равна
$$P_n(k\leq \xi \leq m)  \approx  \Phi\Big(\frac{m-n\mu}{\sqrt{n}\sigma}\Big) - \Phi\Big(\frac{k-n\mu}{\sqrt{n}\sigma}\Big)$$

\subsubsection{Неравенство}
$$\Phi(x) = \frac{1}{\sqrt{2\pi}}\int_{-\infty}^{x}e^{-\frac{t^2}{2}}dt$$

\textbf{ЦПТ}: Пусть есть бесконечная последовательность одинаково распределенных случайных величин сл. вел. $\xi$, имеющих матожидание $\mu$ и дисперсию $\sigma$ тогда
$$\frac{S_n - \mu n}{\sqrt{n}\sigma} \underset{n \to \infty}{\to} N(0,1)$$
где $S_n = \xi_1 + \ldots + \xi_n$

\newpage
\section*{2.4 ЦПТ. Локальная и интегральная т. Муавра-Лапласа}
\subsection*{Практика}
\begin{enumerate}
\item В театре 1600 мест и 2 гардероба. Посетитель выбирает гардероб равновероятно. Сколько в них должно быть мест, чтобы их могло не хватить не чаще раз в месяц.
\item Найти вероятность выйграть в рулетке со ставками на красное и черное, если сделано 200 ставок.

\end{enumerate}
\newpage
\section*{2.4 ЦПТ. Локальная и интегральная т. Муавра-Лапласа}
\subsection*{Домашка}
\begin{enumerate}
\item Имеется 1000 параллепипедов, у каждого из которых длинакаждой стороны может принимать значения
$\frac{1}{2}$ и 1 с вероятностями 0,3 и 0,7 соответственно. Пусть V - суммарный объем этих параллепипедов. 
Оценить вероятность того, что $580< V <605$.

\end{enumerate}

\end{document}

