\documentclass[a4paper, 14pt]{extarticle}

%% Language and font encodings
\usepackage[english, russian]{babel}
\usepackage[utf8]{inputenc}
\usepackage{amsfonts,amssymb,amsmath}

\usepackage[a4paper,top=2cm,bottom=2cm,left=2cm,right=1.5cm,margin=15mm, lmargin=30mm]{geometry}

\begin{document}

\section*{Контрольная 4. Распределения, сходимости.} 
\begin{enumerate}

\item (12) Пусть у нас есть процесс испытаний Бернулли с $n = 10000$ испытаниями и $p$ - вероятностью успеха в этом процессе.
Мы хотим оценить вероятности событий 
\begin{enumerate}
\item (2) Выпало ровно 2000 успехов при $p=0,25$
\item (2) Выпало ровно 7 успехов при $p=0,001$
\item (2) Выпало от 1500 до 2100 успехов при $p=0.2$
\item (2) Выпало больше 6 успехов при $p=0,0005$
\item (2) Выпало ровно 9989 успехов при $p=0.9997$
\item (2) Выпало меньше 1500 успехов при $p=0.5$
\end{enumerate}
с помощью теорем Муавра-Лапласа, т.Пуассона, неравенств Чебышева и Маркова.
Выберите нужную теорему и решите задачу.

\item (3) Последовательность случайных величин $\{\xi_n\}_{n=1}^{\infty}$ распределена по экспоненциальному закону с параметром $\lambda$. 
По ЗБЧ $\frac{\sum \xi_n - \mu}{\sigma \sqrt{n}} \to N(0,1)$. 
Оценить неравенством Бэрри-Эссена скорость сходимости к нормальному распределению.

\item (3)Пусть $G(n,p)$ - случайный граф на n вершинах и вероятностью ребра p. Докажите: 
Докажите, что при вероятности ребра p такой, что $pn^{\frac{5}{4}} \to 0$ при $n \to \infty$ асимптотически почти наверняка нет связных компонент, изоморфных графу-дереву-звезде на 5 вершинах. 

\item (3) Пусть последовательность сл.в. $\{\xi_n\}_{n=1}^{\infty}$ имеют одинаковое 
невырожденное распределение с нулевым средним значением и с конечной дисперсией. 
Найти $D\xi_1$ если
$$\lim_{n\to\infty}P(\frac{\sum_{i=1}^{n}\xi_i}{\sqrt{n}} > 1) = \frac{1}{3}$$

\item (3)Пусть $\xi_n$ принимает значения $n^{-\lambda}$ и $-n^{-\lambda}$ 
с вероятностью 1/2
каждое. Выяснить, при каких значениях $\lambda$ для последовательности $\{\xi_n\}_{n=1}^{\infty}$
выполнена ЦПТ.

\item (3) Пусть $\{\xi_n\}_{n=1}^{\infty}$ - последовательность случайных величин, 
причём $\xi_n$ принимает значения $e^{-\alpha n}$ и $e^{\alpha n}$ с вероятностями $1 - e^{-\beta n}$ и $e^{-\beta n}$
соответственно. При каких значениях $\alpha$ и $\beta$ имеет место сходимость $\xi_n \overset{\mathbb{P}}{\to} \xi$

\item (3) Пусть $\xi_n \overset{\mathbb{P}}{\to} 1$, $\mu_n \overset{\mathbb{P}}{\to} 1$ и 
$\nu_n \overset{\mathbb{P}}{\to} 1$.
Доказать, что $\xi_n - \mu_n \cdot \nu_n \overset{\mathbb{P}}{\to}0$




\end{enumerate}
\end{document}

