\documentclass[a4paper, 14pt]{extarticle}

%% Language and font encodings
\usepackage[english, russian]{babel}
\usepackage[utf8]{inputenc}
\usepackage{amsfonts,amssymb,amsmath}

\usepackage[a4paper,top=2cm,bottom=2cm,left=2cm,right=1.5cm,margin=15mm, lmargin=30mm]{geometry}

\begin{document}

\section*{Контрольная 4. Распределения, сходимости.} 
\begin{enumerate}

\item (9) Пусть у нас есть процесс испытаний Бернулли с $n = 10000$ испытаниями и $p$ - вероятностью успеха в этом процессе.
Мы хотим оценить вероятности событий 
\begin{enumerate}
\item (1.5) Выпало ровно 2000 успехов при $p=0,25$
\item (1.5) Выпало ровно 7 успехов при $p=0,001$
\item (1.5) Выпало от 1500 до 2100 успехов при $p=0.2$
\item (1.5) Выпало больше 6 успехов при $p=0,0005$
\item (1.5) Выпало ровно 9989 успехов при $p=0.9997$
\item (1.5) Выпало меньше 1500 успехов при $p=0.5$
\end{enumerate}
с помощью теорем Муавра-Лапласа или т.Пуассона. Выберите нужную теорему и решите задачу.
Сравните ответ с реальным (посчитайте Сшку питоном или вольфрамом) и посмотрите на какой позиции возникает ошибка. 

\item (3) Последовательность независимых случайных величин $\{\xi_n\}_{n=1}^{\infty}$ распределена по экспоненциальному закону с параметром $\lambda$. 
По ЦПТ $Z_n = \frac{\sum \xi_n - \mathbb{E}(\sum \xi_n)}{\sqrt{\mathbb{D}(\sum \xi_n)}} \to N(0,1)$. 
Оценить неравенством Бэрри-Эссена скорость сходимости к нормальному распределению. 
Найти оценку для n=30, n=300, n=3000.

\item (3) Посчитать асимптотику $C_{kn}^{mn}$ при $n \to \infty$, где $m<k$ и эти параметры константы.

\item (3) Найти характеристическую функцию для гамма-распределения $\Gamma(k,\theta)$:
$$\rho(x) = x^{k-1}\frac{e^{-\frac{x}{\theta}}}{\Gamma(k)\theta^k}~~~~~x\geq0$$

\item (3)Пусть $G(n,p)$ - случайный граф на n вершинах и вероятностью ребра p. Докажите: 
Докажите, что при вероятности ребра p такой, что $pn^{\frac{5}{4}} \to 0$ при $n \to \infty$ асимптотически почти наверняка нет связных компонент, изоморфных графу-дереву-звезде на 5 вершинах. 

\item (3) Пусть последовательность сл.в. $\{\xi_n\}_{n=1}^{\infty}$ имеют одинаковое 
невырожденное распределение с нулевым средним значением и с конечной дисперсией. 
Найти $D\xi_1$ если
$$\lim_{n\to\infty}P(\frac{\sum_{i=1}^{n}\xi_i}{\sqrt{n}} > 1) = \frac{1}{3}$$

\item (3)Пусть $\xi_n$ принимает значения $3^n$ , $-3^n$ и 0 с вероятностями
$3^{-(2n+2)}$, $3^{-(2n+2)}$ и $1 - 2 \cdot 3^{-2n+2}$ соответственно. Выполнен ли ЗБЧ для
последовательности $\xi_n$ 

\item (3) Пусть $\xi_n \overset{\mathbb{P}}{\to} 1$, $\mu_n \overset{\mathbb{P}}{\to} 1$ и 
$\nu_n \overset{\mathbb{P}}{\to} 1$.
Доказать, что $\xi_n + \mu_n \cdot \nu_n \overset{\mathbb{P}}{\to}2$

\item (3)(сложная) Пусть $\{\xi_n\}_{n=1}^{\infty}$ - последовательность случайных величин, 
причём $\xi_n$ принимает значения $e^{-\alpha n}$ и $e^{\alpha n}$ с вероятностями $1 - e^{-\beta n}$ и $e^{-\beta n}$
соответственно. При каких значениях $\alpha$ и $\beta$ имеет место сходимость $\xi_n \overset{\mathbb{P}}{\to} \xi$

\item (3)(сложная) Пусть $\xi_1$ имеет стандартное нормальное распределение. Найти
предельное при $n \to \infty$ распределение величины $\frac{\eta_n}{\zeta_n}$, где
$$\eta_n = \sum_{i=0}^{n-1}\frac{\xi_{2i+1}}{\xi_{2i+2}}~~~~\zeta_n = \sum_{i=1}^{n}\xi_i^2$$



\end{enumerate}
\end{document}

