\documentclass[a4paper, 14pt]{extarticle}

%% Language and font encodings
\usepackage[english, russian]{babel}
\usepackage[utf8]{inputenc}
\usepackage{amsmath, amssymb}

\usepackage[a4paper,top=1cm,bottom=1cm,left=1cm,right=1.5cm,margin=10mm, lmargin=15mm]{geometry}

\title{practice2_7}
\author{Sam Stikhin}
\date{March 2019}

\begin{document}
\section*{2.7 Построение оценок. Метод моментов. Метод максимального правдоподобия}
\section*{Пререквизиты}
\subsection*{Метод моментов}
Пусть у нас есть выборка $X_1, ..., X_n$ из распределения с параметром $\theta$.
Если мы сможем подобрать такую функцию $g$, такую что 
$$\mathbb{E}(g(X_i)) = h(\theta)$$
где $h$ - непрерывна, то оценка методом моментов находится как
$$\hat\theta = h^{-1}(\overline{g(X)})$$

Аналогично происходит, если у нас несколько параметров $\theta = (\mu, \sigma)$. 
Тогда мы составляем систему из нескольких различных функций $g$(по количеству параметров) и решаем ее:
$$\begin{cases}
\mathbb{E}(g_1(X_i)) = h(\theta) \\
\mathbb{E}(g_2(X_i)) = h(\theta) \\
\ldots \\
\mathbb{E}(g_n(X_i)) = h(\theta)
\end{cases}$$

Метод называется методом моментов, так как за функции $g_i$ обычно берутся функции моментов
$$g_k(x) = x^k$$

Так как функции $h$ - непрерывные, оценки полученные методом моментов - всегда состоятельные

\subsection*{Метод максимального правдоподобия}
Пусть у нас есть выборка $\bold X^n = \{X_1, ..., X_n\}$ из распределения $X$ с параметром $\theta$.

Идея: давайте посчитаем вероятность того, что у нас выпала такая последовательность результатов и назовем 
эту величину правдоподобием.

$$Likelihood = L(\bold X^n; \theta) = \begin{cases}
\prod_{i=1}^{n}\rho_{X}(X_i), & \textbf{Для непрерывных распределений} \\
\prod_{i=1}^{n}P(X = X_i), & \textbf{Для дискретных распределений} \\
\end{cases}$$

Распределение $X$ как-то зависит от параметра $\theta$ и мы хотим максимизировать правдоподобие по параметру $\theta$
Итого оценка максимального правдоподобия:

$$\hat\theta = argmax_{\theta}L(\bold X^n; \theta)$$

Понятное дело, что $\theta$ может быть вектором параметров ( Например $\theta = (\mu, \sigma)$ для $\mathcal(\mu, \sigma^2)$). 
Тогда функцию нужно максимаизировать по нескольким параметрам. 

\newpage

\section*{2.7 Построение оценок. Метод моментов. Метод максимального правдоподобия}
\section*{Практика}
\begin{enumerate}
	\item Пусть выборка $X_1, ..., X_n$ порождена распределением $U[0, \theta]$. 
	Оцените параметр $\theta$ с помощью:
	\begin{itemize}
		\item Метода моментов (функцией $g(x) = x$ и $g(x) = x^k$)
		\item Метода максимального правдоподобия
	\end{itemize}
	
	\item Пусть выборка $X_1, ..., X_n$ порождена распределением $\mathcal{N}(\mu, \sigma^2)$. 
	С помощью метода моментов найдите оценки параметров $\hat \mu, \hat\sigma$.
	
	\item Пусть выборка $X_1, ..., X_n$ порождена распределением $\mathcal{N}(0, \sigma^2)$. 
	С помощью метода максимального правдоподобия найдите оценку параметра $\hat\sigma$.
	
	\item Пусть выборка $X_1, ..., X_n$ порождена распределением с плотностью $f(x)$:
	\begin{center}
		$f(x) = \begin{cases}
			\frac{\beta \alpha^{\beta}}{x^{\beta + 1}}, x \ge \alpha\\
			0, x < \alpha
		\end{cases}$
	\end{center}
	Здесь $\alpha > 0$ и $\beta > 0$. С помощью метода максимального правдоподобия постройте оценку параметров $\alpha$ и $\beta$.
	
	
\end{enumerate}
\newpage
\section*{Домашка}
\begin{enumerate}
	\item (1 балл) Используя метод моментов, постройте оценку $\lambda > 1$ по выборке из распределения Пуассона с параметром $\ln \lambda$. 
	
	\item (1 балл) Пусть выборка $X_1, ..., X_n$ порождена распределением с плотностью $f(x)$:
	$$ f(x) = \frac{1}{\sqrt{2\pi}}e^{-\frac{\left(x - a\right)^2}{2}}$$
	Параметр $a$ может принимать значения 1 или 2. Найдите оценку максимального правдоподобия параметра $a$.
	
	\item (1 балл) Пусть выборка $X_1, ..., X_n$ порождена распределением с плотностью $f_{\theta}(x)$: 
	$f_{\theta}(x) = f(x - \theta)$,
	 где функция $f(x)$ имеет единственный максимум в точке $x = 0$. Постройте оценку максимального правдоподобия
	  $\hat \theta$ параметра сдвига $\theta$ по одному наблюдению $X_1$.
	  
	\item (1 балл) Пусть выборка $X_1, ..., X_n$ порождена распределением с плотностью $f(x)$: $$ f(x) = \frac{1}{2\sigma} e^{-\frac{\left|x - \mu\right|}{\sigma}}$$
	Постройте оценку максимального правдоподобия для вектора параметров $\left(\mu, \sigma\right)$.
	
	\item Пусть выборка $X_1, ..., X_n$ порождена равномерным на отрезке $[\theta; 2\theta]$ распределением. Постройте оценку параметра $\theta$:
		\begin{itemize}
			\item (1 балл) Методом моментов.
			\item (1 балл) Методом максимального правдоподобия.
		\end{itemize}
\end{enumerate}
\end{document}
