\documentclass[a4paper, 14pt]{extarticle}

%% Language and font encodings
\usepackage[english, russian]{babel}
\usepackage[utf8]{inputenc}
\usepackage{amsmath}

\usepackage[a4paper,top=2cm,bottom=2cm,left=2cm,right=1.5cm,margin=15mm, lmargin=30mm]{geometry}

\title{practice6}
\author{samstikhin}
\date{September 2018}

\begin{document}

\section*{Контрольная 3. Непрерывные СВ.}
\begin{enumerate}
\item (3б) Вычислить интеграл
$$\int_0^{\infty} \frac{x^{2007}dx}{(1+x)^{2018}}$$
\item (3б) На окружности равновероятно выбираются 3 точки, какова вероятность того, что центр окружности лежит внутри.
\item (3б) На отрезке $[0,1]$ равновероятно выбираются 2 числа. Найдите вероятность, что их сумма не больше 0,9, а произведение не меньше 0.15.
\item (3б) Острый угол прямоугольного треугольника с единичной гипотенузой распределен равномерно от 0 до $\frac{\pi}{2}$. 
Найти функцию распределения и плотность площади треугольника.
\item (3б)Случайная величина $\xi$ имеет непрерывную функцию распределения $F_{\xi}(x)$. 
Найдите функцию распределения случайной величины $\eta=2-\sqrt{\xi}$.
\item (3б)Пусть случайная величина $\xi$ имеет стандартное распределение
Коши. Найти плотности распределения следующих случайных величин:
\begin{enumerate}
\item $\frac{1}{\xi}$
\item $2\xi + 1$
\item $\frac{1}{1+\xi^2}$
\end{enumerate}
\item (3б) Найти коэффициент корреляции между $\xi$ и $\eta = a\xi^2 + c$ , если 
$\xi$ имеет стандартное нормальное распределение $N(0,1)$.
\item (3б) Случайные величины $\xi$ и $\eta$
независимы и имеют экспоненциальное распределение с параметром
$\lambda>0$. Найдите математическое ожидание и дисперсию случайной величины
$|\xi - \eta|$.


\item (3б)Совместное распределение случайных величин $\xi$ и $\eta$ имеет плотность 
$\rho = e^{-(x+y)}$ (x,y > 0). 
\begin{enumerate}
\item (1)$E_{\xi}$, $E_{\eta}$
\item (1)$D_{\xi}$, $D_{\eta}$
\item (1)$cov(\xi,\eta)$
\end{enumerate}

\item (3б)Найти $E(\xi|\eta)$, если совместная плотность случайного вектора
$(\xi, \eta)$ равна:
	$$\rho_{\xi,\eta}(x,y) =\left\{
	\begin{array}{cc}
	ye^{-x}+e^{-2x}, & 0\leq x, 0\leq y\leq 1 \\
	0, & otherwise
\end{array}\right.$$





\end{enumerate}
\end{document}

