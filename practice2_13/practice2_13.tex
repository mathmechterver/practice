\documentclass[a4paper, 14pt]{extarticle}

%% Language and font encodings
\usepackage[english, russian]{babel}
\usepackage[utf8]{inputenc}
\usepackage{amsmath}

\usepackage[a4paper,top=1cm,bottom=1cm,left=1cm,right=1cm, margin=10mm, lmargin=10mm]{geometry}
\title{practice2old}
\author{samstikhin}
\date{September 2018}

\begin{document}
\subsection*{Непрерывные марковские цепи. Диффуры.}
\begin{enumerate}
\item Пусть некий программист пытается решить сложную задачу, а время решения задачи
распределено по показательному закону с параметром $\lambda$. Найдите вероятность того, что программист
решит задачу.
\item Пусть имеется 2 программиста, наперегонки решающие сложную задачу, а время реше-
ния задачи распределено по показательному закону с параметром $\lambda_i$ у каждого. Найдите вероятность
того, что задача будет решена первым программистом.
\item Пусть некий программист пытается решить сложную задачу до того, как его уволят, а
время решения задачи распределено по показательному закону с параметром $\lambda$, также имеется фирма,
терпение которой распределено по показательному закону с параметром $\mu$. Найдите вероятность того,
что программист решит задачу раньше, чем будет уволен.

\item Предположим, что временные интервалы между последовательными приходами в магазин 
посетителей — независимые случайные величины, имеющие показательное распределение 
с параметром $\lambda$. Постройте для состояний из $\{0, 1, 2, \ldots \}$ (число посетителей) 
марковскую цепь с непрерывным временем. 
Найдите вероятность того, что к моменту времени 5 придут ровно k посетителей.

\item Предположим, что временные интервалы между последовательными приходами в мага-
зин посетителей — независимые случайные величины, имеющие показательное распределение с пара-
метром $\lambda$ для женщин и $\mu$ для мужчин. Найдите распределение общего числа посетителей. Найдите
вероятность того, что первые три посетителя будут мужчины. Найдите вероятность, что после того,
как зашло ровно 421 покупателей, 422 -й покупатель зашел в течение минуты. Тот же вопрос, если
423 -й покупатель зашел меньше через 2 минуты после 421 -го.

\item Ласт, Самунь, Бэй и Корнев принимают экзамен по питону в группе гуманитариев. В каждый момент времени каждый из них разговаривает с не более чем одним студентом. Если очередной студент приходит в момент, когда все преподаватели заняты, студент отправляется на пересдачу. Гуманитарии просты и наивны, поэтому интервалы времени между последовательно приходящими студентами можно считать независимыми случайными величинами, распределёнными по показательному закону с параметром $\lambda$. Интервалы времени, в течение которых студент отвечает преподавателю будем полагать независимыми и имеющими экспоненциальное распределение с параметром $\mu$. Предполагается, что студентов бесконечно много.
\begin{itemize}
\item Постройте марковскую цепь с непрерывным временем, описывающую данный процесс.
\item Найдите распределение количества занятых преподавателей в момент времени $t$ от начала экзамена. 
\item Найдите стационарное распределение данного процесса.
\end{itemize}

\end{enumerate}
\newpage
\subsection*{Непрерывные марковские цепи. Диффуры.}
\begin{enumerate}
\item (1) Машины проезжают мимо поста ГИБДД по экспоненциальному закону с параметром $\frac{1}{4}$. На посту ГИБДД двое, если хотя бы один из них не занят беседой, он тормозит первую же машину. Беседа занимает в среднем время $10$, время распределено по экспоненциальному закону, для беседы достаточно одного постового. Пусть в момент времени 0 постовые оба не заняты, найдите вероятность того, что они оба не будут заняты в момент времени t.
\item (1) Пусть имеется 5 серверов, которые пытаются независимо друг от друга найти запись в базе данных, время поиска предположим распределенным по показательному закону с параметром $\lambda_i$ у каждого. Найдите вероятность того, что первый отклик поступит от сервера 1.
\item (2) Некий вирус может находиться в одном из $m$ штаммов, в каждом — случайное время распределенное по экспоненциальному закону с параметром $\lambda_i$, после чего с равными вероятностями мутирует в один из штаммов. Раз в секунду номер штамма этого вируса записывается отдельной строкой в лог-файл. Найдите марковскую цепь с дискретным временем, описывающую последнюю строку лога.
\item (2) Решите задачу про преподавателей с очередью, если очередь длины 1. (+1 балл) длины $k$
\item (2) Пусть имеется 5 программистов, которые пытаются независимо друг от друга решить
сложную задачу до того, как их всех уволят, а время решения задачи распределено по показательному
закону с параметром $\lambda_i$ у каждого. Есть заказчик, терпение которого распределено по показательному
закону с параметром $\nu$, и фирма, терпение которой (пока есть заказчик) распределено по показатель-
ному закону с параметром $\mu$, без заказчика ее терпение равно нулю, а увольняет она сразу и всех.
Найдите вероятность того, что а) задача будет решена, б) задача будет решена первым программистом.
\item (3) В бюро приходят клиенты по закону Пуассона с параметром $\lambda$, немедленно попадая к
одному из клерков (будем считать что их бесконечно много). На получение нужной справки затрачи-
вается распреденное по экспоненциальному закону время (в среднем $\tau$). С вероятностью $q$ справка
делается неправильно, в этом случае клиент не уходит, а идет к директору, и, при необходимости подождав
 в очереди, общается с директором распреденное по экспоненциальному закону время (в среднем
$\theta$ ). Покажите, что при $q\theta < \lambda$ у марковской цепи все вершины которой пронумерованы $(r, s)$ ( $r$ - число
 занятых клерков, $s$ - число клиентов в очереди) стационарное распределение можно представитьв виде $\frac{e^{-\alpha}\alpha^r}{r!}(1-\alpha)\alpha^s$ .
\end{enumerate}

\end{document}
