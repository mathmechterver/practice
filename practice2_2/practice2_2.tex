\documentclass[a4paper, 14pt]{extarticle}

%% Language and font encodings
\usepackage[english, russian]{babel}
\usepackage[utf8]{inputenc}

\usepackage[a4paper,top=1cm,bottom=1cm,left=1cm,right=1.5cm,margin=10mm, lmargin=15mm]{geometry}

\usepackage{amsfonts,amssymb,amsmath}
\usepackage{nopageno, comment}
\usepackage{cmap}
\usepackage{ifthen}
\usepackage{indentfirst}
\usepackage{float}
\usepackage{tikz}
\usepackage{wrapfig}

\title{practice2_2}
\author{Sam Stikhin}
\date{February 2019}

\begin{document}

\section*{2.2 Неравенства Маркова/Чебышева. ЗБЧ}
\subsection*{Пререквизиты}
Выборочное среднее:
$$\overline{X} = \frac{X_1 + \ldots + X_n}{n}$$

\textbf{Неравенство Маркова}: Пусть $\xi$ - неотрицательная, $E\xi$ - существует и конечно, $a>0$ тогда
$$P(\phi(\xi) \geq \phi(a)) \leq \frac{E\phi(\xi)}{\phi(a)}$$

Для неотрицательной, монотонно неубывающей ф-ии $\phi$
$$P(\xi \geq a) \leq \frac{E\xi}{a}$$

\textbf{Неравенство Чебышева}: Пусть $E\xi$ и $D\xi$ - существуют и конечны, $a>0$ тогда
$$P(|\xi - E\xi| \geq a) \leq \frac{D\xi}{a^2}$$

\textbf{Закон Больших Чисел}: Пусть $E\xi$ и $D\xi$ - существуют и конечны, $a>0$ тогда
$$P\left(\left|\frac{X_1 + \ldots + X_n}{n} - EX_1\right| \geq a\right) \leq \frac{DX_1}{na^2}$$

\textbf{Неравенство Хёфдинга} Если случайные величины $X_1,\dots,X_n$ принимают значения из некоторого отрезка длины $d$, имеют место неравенства Хёфдинга: для всех $c>0$
$$P({|\overline{X}-\mathbb{E}{X}_1|\geq c})\leq 2e^{-2nc^2/d^2}$$

Нам хватит и такого:
$$P({\overline{X}-\mathbb{E}{X}_1\geq c})\leq 2e^{-2nc^2/d^2}.$$

\newpage
\section*{2.2 Неравенства Маркова/Чебышева. ЗБЧ}
\subsection*{Практика}
\begin{enumerate}
\item В лотерее на выигрыши уходит $49\%$ от стоимости проданных билетов. Каждый билет стоит 200 рублей. Оцените сверху вероятность выиграть 10000 рублей или больше.
\item Правильная игральная кость подбрасывается 1000 раз. С помощью неравенства Чебышёва оцените вероятность того, что среднее арифметическое числа выпавших очков отклонится от математического ожидания по абсолютной величине не более чем на 0,3. 
\item Оцените вероятность того, что в результате подбрасывания игральной кости в течение 500 раз относительная частота появления на верхней грани шестерки отклонится от вероятности этого события (по абсолютной величине) не более чем на 0,05. 

%\item Четыре человека пришли к врачу. Время приема одного человека распределено равномерно и составляет от одной до четырех минут. При помощи неравенства Хефдинга и границы Чернова оцените вероятность того, что все четыре человека будут осмотрены более чем за 12 минут.

\item Время отклика сервера равномерно распределено на отрезке $\left[0.375; 0.45\right]$. При помощи неравенств Маркова, Хефдинга и границы Чернова оцените вероятность того, что $1000$ независимых в совокупности запросов будут обработаны суммарно быстрее, чем за $413$ секунд.

\item Пусть $G(n,p)$ - случайный граф на n вершинах и вероятностью ребра p. Докажите: 
\begin{enumerate}
    \item что при вероятности ребра $p$ такой, что $pn^{\frac{2}{3}}\to 0$ при $n\to +\infty$ в $G(n,p)$ асимптотически почти наверняка нет подграфов, изоморфных графу $K_4$.
    \item что при вероятности ребра $p$ такой, что $pn^{\frac{2}{3}}\to +\infty$ при $n\to +\infty$ в $G(n,p)$ асимптотически почти наверняка существует подграф, изоморфных графу $K_4$.
    \item Докажите, что при вероятности ребра p такой, что $pn^{\frac{5}{4}} \to 0$ при $n \to \infty$ асимптотически почти наверняка нет связных компонент, изоморфных графу-дереву-звезде на 5 вершинах. 
\end{enumerate}

\end{enumerate}
\newpage
\subsection*{Домашка}
\begin{enumerate}
\item Конспект по терверу объёмом 500 страниц содержит 50 опечаток (каждая из них равновероятно находится на одной из страниц). Оцените с помощью неравенств Маркова вероятность того, что на некоторой странице содержится 
\begin{enumerate}
    \item (1)не менее 3 опечаток.
    \item (1)0 опечаток.
\end{enumerate}
\item (1) Великий Морской путь сулит значительную прибыль. Однако из-за пиратов в среднем груз одного из 20 кораблей не достигает порта назначения. Чтобы скомпенсировать убытки n коммерсантов создали фонд, в который складывают $6\%$ прибыли с каждого корабля, уходящего в плавание. При помощи неравенства Чебышева/Маркова найдите наименьшее n, при котором вероятность, что фонд не сможет возместить убытки не превышает 0.05.
\item (1) Передаётся слово длины n. В каждой позиции с вероятностью p независимо друг от друга происходит ошибка. При этом вероятность ошибки зависит от длины слова: $p = o(\frac{1}{n})$. Докажите, что при $n \to \infty$ ошибки происходят с вероятностью 0.
\item Пусть $G(n,p)$ - случайный граф на n вершинах и вероятностью ребра p. Докажите: 
\begin{enumerate}
    \item (1)что при вероятности ребра $p$ такой, что $pn \to 0$ при $n\to +\infty$ в $G(n,p)$ асимптотически почти наверняка нет подграфов, изоморфных графу $K_3$.
    \item (1)что при вероятности ребра $p$ такой, что $pn \to +\infty$ при $n\to +\infty$ в $G(n,p)$ асимптотически почти наверняка существует подграф, изоморфный графу $K_3$.
\end{enumerate}
\item (Усложнение 1 задачи.) Оцените вероятность того, что на некоторой странице содержится не менее 3 опечаток 
\begin{enumerate}
    \item (1)с помощью Неравенства Хефдинга 
    \item (1)с помощью границы Чернова
    \item (1)c помощью границы моментов
\end{enumerate}

\end{enumerate}

\end{document}

