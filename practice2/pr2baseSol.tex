\documentclass[a4paper, 14pt]{extarticle}

%% Language and font encodings
\usepackage[english, russian]{babel}
\usepackage[utf8]{inputenc}

\usepackage[a4paper,top=2cm,bottom=2cm,left=2cm,right=1.5cm,margin=15mm, lmargin=30mm]{geometry}

\title{pr2baseSol}
\author{samstikhin }
\date{September 2018}

\begin{document}

\section*{Классическая вероятность. Схема Бернулли.}
\subsection*{Базовый}
\begin{enumerate}
\item  Из 28 костей домино случайно выбираются две.
Найти вероятность  того, что из них можно 
составить <<цепочку>> согласно правилам игры.

\textbf{Решение:}
Всего способов достать 2 доминошки: $C_{28}^2$. Нужно найти множество хороших пар: у которых есть общая цифра. Разобьем их на 2 множества: в котором есть один дубль и в котором нет дублей. Два дубля быть не может, значит мы не потеряли ни одной пары. \newline
В множестве, в котором есть дубли мы сначала берем один из 7 дублей и для каждого находим 6 подходящих доминошек с нужной цифрой. Итого: 7*6 хороших пар доминошек с дублями.
\newline
В множестве, где нет дублей берем одну из оставшихся 21 доминошек и ищем к ней 10 соседей не дублей (по 5 для каждой цифры, потому что дубли уже убрали). Осталось поделить пополам, потому что все пары мы посчитали по 2 раза (пример: в начале взяли доминошку 1-2, потом 2-3 и наоборот, сначала 2-3, потом 1-2). Итог: $21*10/2=105$.
\textbf{Ответ:} $105+42 = 147$.

\item Ребенок играет с десятью буквами разрезной азбуки: А, А, А,
Е, И, К, М, М, Т, Т. Какова вероятность того, что при случайном
расположении букв в ряд он получит слово <<МАТЕМАТИКА>>?

\textbf{Решение:}
Всего перестановок кубиков $10!$. Посчитаем в каких перестановках слова не меняются. Буквы "М", "T" и "А" можно менять между собой и получать исходное слово. Значит слово <<МАТЕМАТИКА>> можно получить $2!2!3!$ способами. Вероятность: $$\frac{2!2!3!}{10!}$$ (перевернутый полиномиальный коэффициент).

\item Охотник стреляет в лося с расстояния 100 м и 
	попадает в него с вероятностью 0.5. Если при первом выстреле
	попадания нет, то охотник стреляет второй раз, но с 
	расстояния 150 м. Если нет попадания и в этом случае, то охотник
	стреляет третий раз, причем в момент выстрела расстояние до
	лося равно 200 м. Считая, что вероятность попадания 
	обратно пропорциональна квадрату расстояния, определить 
	вероятность попадания в лося.

\textbf{Решение:}
    Вероятность обратно пропорциональна квадрату расстояния значит: $p = \frac{k}{r^2}$, где $r$ - расстояние до лося. Для $p = 0.5$, $r=100$: $k=5000$. Значит 
    $p_{150} = \frac{5000}{150^2} = \frac{2}{9}$ и 
    $p_{200} = \frac{5000}{200^2} = \frac{1}{8}$
    \newline
    Теперь найдем вероятность попасть в лося с трех попыток:
    $$P = p_{100} + q_{100}p_{150} + q_{100}q_{150}p_{200} = \frac{1}{2} + \frac{1}{2}\frac{2}{9} + \frac{1}{2}\frac{7}{9}\frac{1}{8} = \frac{72+16+7}{9*16} = \frac{95}{144}$$

\item В урне находится $m$ шаров, из которых $m_1$ белых и $m_2$ черных
($m_1 + m_2 = m$). Производится $n$ извлечений одного шара с возвращением его (после определения его цвета) обратно в урну. Найдите
вероятность того, что ровно $r$ раз из $n$ будет извлечен белый шар.

\textbf{Решение:}
В явном виде схема Бернулли. $p =\frac{m_1}{m}$ $q =\frac{m_2}{m}$
$$P = C_n^r p^r q^{n-r} = P = C_n^r \Big(\frac{m_1}{m}\Big)^r \Big(\frac{m_2}{m}\Big)^{n-r}$$

\item Вероятность отказа каждого прибора при испытании
	равна 0,2. Сколько таких приборов нужно испытать, чтобы с
	вероятностью не менее 0,9 получить не меньше трех отказов?

\textbf{Решение:} 
Пойдем от обратного и найдем такое количество приборов, что веротноять встретить меньше трех отказов была меньше 0.1. Пусть провели n испытаний, тогда найдем вероятности, при которых произойдет 0 отказов, 1 отказ, 2 отказа. Вероятность отказа обозначим за $p$ (успех в схеме Бернулли)
$P_0 = (1-p)^n$, $P_1 = C_n^1p(1-p)^n$, $P_2 = C_n^2 p^2 (1-p)^{n-2}$. Получаем неравенство:
$$0.8^n + n*0.2*0.8^{n-1} + \frac{n(n-1)}{2}*0.2*0.8^{n-2} \le 0.1$$
Это неравенство (спасибо Wolfram) выполняется для $n\ge 25$.

\item (Задача Стефана Банаха) В двух спичечных коробках
имеется по $n$ спичек. На каждом шаге наугад выбирается коробок, и из него удаляется (используется) одна спичка. Найдите вероятность того, что в момент, когда один из коробков опустеет, в другом останется $k$ спичек.


\textbf{Решение:}
Представим эксперимент как ленту с клетками. <<0>> будем обозначать взятие спички из одного коробка, <<1>> из другого. Нам нужно, чтобы одновременно произошло 2 события: \begin{enumerate} 
\item взяли последнюю спичку из одного коробка
\item в другом коробке осталось $k$ спичек.
\end{enumerate} 
Это значит, что в хвосте этой ленты будет $k$ <<1>>, а перед ними <<0>> (если бы нолика не было, то под наш случай подходил бы вариант "сначала забрать все спички из одного коробка", но так нельзя). 
Итого у нас $k+1$ зарезервированных мест на ленте. А для оставшихся мы можем использовать схему бернулли. \newline
Всего экспериментов: $2n - k - 1$, количество успехов (0-ков): $n-1$, неуспехов: $n-k$, вероятность успеха $p=\frac{1}{2}$.  
$$C_{2n-k-1}^{n-1}p^{n-1}(1-p)^{n-k} = \frac{1}{2^{2n-k-1}}C_{2n-k-1}^{n-1}$$
Еще нужно домножить на $\frac{1}{2}$ - вероятность выпадения последнего 0-ка (так как это событие может и не произойти, а вот дальше умножать не надо, так как у нас нет выбора и мы берем оставшиеся спички из последнего коробка). \newline 
Так как нам неважно в каком из коробков спички кончились раньше, то мы можем поменять 0 на 1 и наоборот и получить еще столько же подходящих исходов. Поэтому ответ нужно умножить на 2.

\textbf{Ответ:}$$ \frac{1}{2^{n-k-1}}C_{2n-k-1}^{n-1}$$


\item В корзине лежит $n$ шариков. В ходе эксперимента с равными вероятностями вытаскивают шарики из корзины и кладут обратно. Эксперимент заканчивается, когда один из шариков достали $k$ раз. Определить вероятность того, что для этого
придется производить $m<2k$ вытаскиваний.

\textbf{Решение:}
Эксперимент заканчивается, когда один конкретный шарик достали $k$ раз. Значит последний результат у нас определен и не будет учитываться в схеме Бернулли. Составим схему Бернулли для вытаскивания определенного шарика $k$ раз: $m-1$ вытаскиваний, $k-1$ успехов, $p=\frac{1}{n}$ вероятность вытащить каждый шарик. И не забудем домножить на $p$ - выпадение последнего успеха.
$$pC_{m-1}^{k-1}p^{k-1}(1-p)^{m-k}$$
Нам нужно посчитать все успехи для диапазона $m<2k$. Можем просто просуммировать эти вероятности, так как эксперименты не пересекаются, ведь последний эксперимент всегда стоит на разных позициях. Заметим, что при $m<k$ $P=0$, нельзя получить $k$ успехов в меньше, чем $k$ экспериментах. Поэтому суммируем от $m=k$, до $m=2k-1$:
$$P = \sum_{m=k}^{2k-1} pC_{m-1}^{k-1}p^{k-1}(1-p)^{m-k} = \sum_{m=k}^{2k-1}\frac{1}{n^m}C_{m-1}^{k-1}(n-1)^{m-k}$$
За существенное сокращение этой формулы (как минимум убрать сумму) даю 2б:)
\end{enumerate}


\end{document}

