\documentclass[a4paper, 14pt]{extarticle}

%% Language and font encodings
\usepackage[english, russian]{babel}
\usepackage[utf8]{inputenc}

\usepackage[a4paper,top=1cm,bottom=1cm,left=1cm,right=1.5cm,margin=10mm, lmargin=15mm]{geometry}

\usepackage{amsfonts,amssymb,amsmath}
\usepackage{nopageno, comment}
\usepackage{cmap}
\usepackage{ifthen}
\usepackage{indentfirst}
\usepackage{float}
\usepackage{tikz}
\usepackage{wrapfig}

\title{practice2_2}
\author{Sam Stikhin}
\date{February 2019}

\begin{document}

\section*{2.3 Сходимости случайных величин.}
\subsection*{Пререквизиты}
\subsubsection*{Леммы Борелля-Кантелли}
\textbf{1-я}

Пусть есть последовательность (необязательно независимых) событий $\{A_i\}_{i=1}^{\infty}$

Обозначим событие $A = \bigcap_{n=1}^{\infty}\bigcup_{i=n}^{\infty}A_i$

Пусть ряд сходится $\sum_{i=1}^{\infty}P(A_i) < \infty$

Тогда $P(A) = 0$

\textbf{2-я}

Пусть есть последовательность совместно независимых событий $\{A_i\}_{i=1}^{\infty}$

Обозначим событие $A = \bigcap_{n=1}^{\infty}\bigcup_{i=n}^{\infty}A_i$

Пусть ряд расходится $\sum_{i=1}^{\infty}P(A_i) \to \infty$

Тогда $P(A) = 1$
\subsubsection*{Сходимости}
\textbf{Почти наверное}
$$\xi_n \overset{\textrm{п.н.}}{\to} \xi$$

Если

$$P(\{ \omega \in \Omega : \xi_n(\omega) \underset{n \to \infty}{\to} \xi(\omega) \}) = 1$$ 

или

$$P(\{ \omega \in \Omega : \xi_n(\omega) \underset{n \to \infty}{\not\to} \xi(\omega) \}) = 0$$ 

или эквивалентно

$$\forall \varepsilon > 0 : P(\{ \omega \in \Omega : \sup_{k \geq n} |\xi_k(\omega)-\xi(\omega)| > \varepsilon \}) \underset{n \to \infty}{\to} 0$$

\textbf{По вероятности}
$$\xi_n \overset{\mathbb{P}}{\to} \xi$$ 

Если

$$\forall \varepsilon > 0 : P(\{\omega \in \Omega : |\xi_n(\omega)-\xi(\omega)| > \varepsilon \}) \underset{n \to \infty}{\to} 0$$

\subsubsection*{Усиленный Закон больших чисел}
Говорят, что последовательность случайных величин (возможно разнораспределенных) $\{\xi_i\}_{i=1}^{\infty}$ с конечными первыми моментами удовлетворяет закону больших чисел (ЗБЧ), если
$$\frac{\sum_{i=1}^n\xi_i}{n} - \frac{\sum_{i=1}^n\mathbb{E}\xi_i}{n} \overset{\mathbb{P}}{\to} 0$$
	
	  
\newpage
\section*{2.3 Сходимости случайных величин.}
\subsection*{Практика}
\begin{enumerate}

\item Точка путешествует по целым числам. Каждый раз она шагает на $1$, с вероятностью $p$ — вправо, с вероятностью $1-p$ — влево. 
Докажите, что при $p \neq \frac{1}{2}$ вероятность того, что она вернется в исходное положение бесконечное число раз равна $0$.

\item Пусть $\xi_n \to \xi$ п.н и $g(x)$ - непрерывная функция. Докажите, что $g(\xi_n) \to g(\xi)$ п.н.

\item Пусть $(\xi_n - \xi)^2 \to 0$ п.н. Доказать, что $\xi_n \to \xi$ п.н.

\item Пусть $\xi_n \overset{\mathbb{P}}{\to} \xi$ и $g(x)$ - непрерывная, дифференцируемая и монотонно возрастающая функция. Докажите, что $g(\xi_n) \overset{\mathbb{P}}{\to} g(\xi)$.

\item Пусть $\xi_n$ принимает значения $2^n$, $- 2^n$ и $0$ с вероятностями $2^{-(2n+1)}$, $2^{-(2n+1)}$ и $1 - 2^{-2n}$ соответственно. 
Докажите, что для $\xi_n$ выполняется ЗБЧ.


\end{enumerate}
\newpage
\subsection*{Домашка}
\begin{enumerate}
\item (2) Точка путешествует по целым числам. Каждый раз она шагает на $1$, с вероятностью $p$ — вправо, с вероятностью $1-p$ — влево. 
Докажите, что при $p = \frac{1}{2}$ вероятность того, что она вернется в исходное положение бесконечное число раз равна $1$.


\item Пусть $\xi_n \overset{\textrm{п.н.}}{\to} 1$ и $\mu_n \overset{\textrm{п.н.}}{\to} 1$. 
Тогда 
\begin{enumerate}
\item (1)$\xi_n + \mu_n \overset{\textrm{п.н.}}{\to} 2$
\item (1)$\xi_n\mu_n \overset{\textrm{п.н.}}{\to} 1$
\end{enumerate}

\item (2)Пусть $\xi_n$ - последовательность независимых и равномерно распределенных на [0,1] случайных величин. 
Найдите распределение случайной величины $m_n=\min (\xi_1,\ldots,\xi_n)$. Докажите, что $m_n$ стремится почти наверное к 0

\item (1) Пусть $\xi_n$ принимает значения $n$, $-n$ и $0$ с вероятностями $\frac{1}{2n^2}$, $\frac{1}{2n^2}$ и $1 - \frac{1}{n^2}$.
Выполнен ли для этой последовательности закон больших чисел? 

\item (1)Пусть $\xi_n \overset{\textrm{п.н.}}{\to} 1$ и $\mu_n \overset{\textrm{п.н.}}{\to} 1$. 
Тогда $\frac{1}{\xi_n + \mu_n} \overset{\mathbb{P}}{\to} \frac{1}{2}$

\item (4) Точка начинает путешествие по целочисленной решетке в $\mathbb{R}^2$ с вероятностями пойти в любое направление $\neq \frac{1}{4}$.
Докажите что с вероятностью 1, точка когда-нибудь вернется в начальную точку.



\end{enumerate}

\end{document}
