\documentclass[a4paper, 14pt]{extarticle}

%% Language and font encodings
\usepackage[english, russian]{babel}
\usepackage[utf8]{inputenc}
\usepackage{amsmath, amssymb}

\usepackage[a4paper,top=1cm,bottom=1cm,left=1cm,right=1.5cm,margin=10mm, lmargin=15mm]{geometry}

\title{practice2_7}
\author{Sam Stikhin}
\date{March 2019}

\begin{document}

\section*{2.8 Построение оценок. Метод моментов. Метод максимального правдоподобия}
\section{Практика}
\begin{enumerate}
	\item Используя метод моментов, оцените параметр $\theta$ 	равномерного распределения на отрезке:
	\begin{itemize}
		\item $\left[\theta - 1; \theta + 1\right]$, $\theta \in R$
		\item $\left[-\theta; \theta\right]$, $\theta > 0$.
	\end{itemize}
	
	\item Пусть выборка $X_1, ..., X_n$ порождена распределением с плотностью $f(x)$:
	\begin{center}
		$f(x) = \begin{cases}
			\frac{\beta \alpha^{\beta}}{x^{\beta + 1}}, x \ge \alpha\\
			0, x < \alpha
		\end{cases}$
	\end{center}
	Здесь $\alpha > 0$ и $\beta > 0$. С помощью метода максимального правдоподобия постройте оценку параметров $\alpha$ и $\beta$.
	
	\item Пусть выборка $X_1, ..., X_n$ порождена распределением с плотностью $f(x)$:
	\begin{center}
		$f(x) = \begin{cases}
			\alpha e^{-\alpha\left(x - \beta\right)}, x \ge \beta\\
			0, x < \beta
		\end{cases}$
	\end{center}
	Здесь $\alpha > 0$. Постройте оценки параметров $\alpha$ и $\beta$ с помощью метода моментов и метода максимального правдоподобия.
	
	\item Пусть выборка $X_1, ..., X_n$ порождена распределением:
	\begin{center}
		$\begin{cases}
			P(X_i = 1) = p_1\\
			P(X_i = 2) = p_2\\
			P(X_i = 3) = p_3\\
		\end{cases}$
	\end{center}
	$p_1 + p_2 + p_3 = 1$. Постройте оценку параметров $p_1, p_2, p_3$ методом максимального правдоподобия.
\end{enumerate}
\newpage

\section{Домашка 2}
\begin{enumerate}
	\item (2 балла, по одному на каждый метод)
	Найти оценку максимального правдоподобия и метода моментов 
	параметра $p \in (0, 1)$ геометрического распределения.
	\item (1 балл) Пусть дана выборка из распределения с плотностью
	$$f_{\alpha}(y) = 
	\begin{cases}
		3y^2\alpha^{-3}e^{-(\frac{y}{\alpha})^3} & y\geq 0\\
		0 & y < 0
	\end{cases}$$
	Построить оценку параметра $\alpha > 0$ с помощью метода моментов используя k-ый момент $g(y) = y^k$
	\item (1 балл) Пусть выборка $X_1, ..., X_n$ порождена распределением с плотностью $f(x)$: $$ f(x) = \frac{1}{2\sigma} e^{-\frac{\left|x - \mu\right|}{\sigma}}$$
	Постройте оценку методом моментов для вектора параметров $\left(\mu, \sigma\right)$.
	\item (2 балла, по одному на каждый метод) 
	Постройте оценки параметров с помощью метода моментов и метода правдоподобия для 
	Гамма-распределения с двумя параметрами:
	$$f_{k, \theta}(y) = 
	\begin{cases}
		x^{k-1}\frac{e^{-\frac{x}{\theta}}}{\theta^k\Gamma(k)} & x\geq 0\\
		0 & x < 0
	\end{cases}$$
	\item (2 баллa) Пусть выборка $X_1, ..., X_n$ порождена распределением:
	\begin{center}
		$\begin{cases}
			P(X_i = 1) = p_1\\
			P(X_i = 2) = p_2\\
			P(X_i = 3) = p_3\\
			P(X_i = 4) = p_4\\
		\end{cases}$
	\end{center}
	$p_1 + p_2 + p_3 + p_4 = 1$. Постройте оценку параметров $p_1, p_2, p_3, p_4$ методом максимального правдоподобия.
	\item (3 балла) Пусть выборка $X_1, ..., X_n$ порождена распределением Коши. Доказать, что медиана - оценка метода максимального прадободобия. 
	(P.S. - Не забудьте, что у распределения Коши не существует матожидания:))
		
	
\end{enumerate}
\end{document}
