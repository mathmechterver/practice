\documentclass[a4paper, 14pt]{extarticle}

%% Language and font encodings
\usepackage[english, russian]{babel}
\usepackage[utf8]{inputenc}

\usepackage[a4paper,top=1cm,bottom=1cm,left=1cm,right=1.5cm,margin=10mm, lmargin=15mm]{geometry}

\usepackage{amsfonts,amssymb,amsmath}
\usepackage{nopageno, comment}
\usepackage{cmap}
\usepackage{ifthen}
\usepackage{indentfirst}
\usepackage{float}
\usepackage{tikz}
\usepackage{wrapfig}


\begin{document}

\section{Тервер Мин}
\subsection{Комбинаторика}

$$C_n^k = \frac{n!}{(n-k)!k!}$$

$$C_n^k \sim \frac{n^k}{k!}$$

$$n! \sim \sqrt{2\pi n}\Big(\frac{n}{e}\Big)^n$$

\subsection{Классическая вероятность}

$$P(A) = \frac{m}{n} = \frac{\sum_{\omega \in \Omega}1_{\omega \in A}}{|\Omega|} = \frac{\textrm{Кол-во хороших элементарных событий}}{\textrm{Кол-во всех элементарных событий}}$$

$A$ и $B$ - несовместны $\Leftrightarrow P(AB) = 0$

$A$ и $B$ - независимы $\Leftrightarrow P(AB) = P(A)P(B)$


\subsection{Геометрическая вероятность}

$$P(A) = \frac{\textrm{Площадь покрывающая событие А}}{\textrm{Вся площадь}}$$

\subsection{Колмогоровская вероятность}

$$F_{\xi}(x) = P(\xi < x)$$

$$\rho_{\xi}(x) = \frac{dF_\xi(x)}{dx}$$

$$F_{\xi}(x) = \int_{\infty}^{\infty}\rho(x)dx$$

\subsection{Свойства матожидания и дисперсии}

$$\mathbb{E}\xi = \sum_{i=1}^{n} x_i p(\xi=x_i)$$

$$\mathbb{E}\xi = \int_{-\infty}^{\infty} x p(\xi=x)dx$$

$$\mathbb{E}(\alpha \xi + \beta \mu) = \alpha \mathbb{E}\xi + \beta \mathbb{E}\mu$$

$$\mathbb{D}\xi = \mathbb{E}\xi^2 - (\mathbb{E}\xi)^2$$

$$\mathbb{E}(\alpha \xi + \beta \mu) = \alpha \mathbb{E}\xi + \beta \mathbb{E}\mu$$

$$cov(\xi, \mu) = \mathbb{E}\xi\mu - \mathbb{E}\xi\mathbb{E}\mu$$

$$\mathbb{D}(\alpha\xi) = \alpha^2 \mathbb{D}\xi $$

$$\mathbb{D}(\xi + \mu) = \mathbb{D}\xi + \mathbb{D}\mu + cov(\xi, \mu)$$

\subsubsection{Для независимых}

$$\mathbb{E}\xi\mu = \mathbb{E}\xi\mathbb{E}\mu$$

$$cov(\xi, \mu) = 0$$

$$\mathbb{D}(\xi + \mu) = \mathbb{D}\xi + \mathbb{D}\mu$$



\section{Условная вероятность}

$$P(A|B) = \frac{P(AB)}{P(B)}$$

$$P(A|B) = \frac{P(B|A)P(A)}{P(B)}$$

Если $H_i$ - попарно несовместны.
$$P(A) = \sum_{i=1}^{n}P(A|H_i)P(H_i)$$

$$\mathbb{E}(\xi|A) = \frac{\mathbb{E}(\xi1_{A})}{P(A)}$$

$$E\xi = \sum_{i=1}^{n}\mathbb{E}(\xi|H_i)P(H_i)$$

\section{Биноминальное распределение}

$$P(\xi = k) = C_n^kp^kq^{n-k}$$

$$\mathbb{E}\xi = np~~~\mathbb{D}\xi = npq$$


\section{Распределение Пуассона}

$$C_n^kp^kq^{n-k} \sim \frac{(np)^{-k}e^{-np}}{k!} = \frac{\lambda^{-k}e^{-\lambda}}{k!}$$

$$\mathbb{E}\xi = \lambda~~~\mathbb{D}\xi = \lambda$$


\section{Экспоненциальное распределение}

$$F_{\xi}(x) = 1 - e^{-\lambda x}$$

$$\rho_{\xi}(x) = \lambda e^{-\lambda x}$$

$$\mathbb{E}\xi = \lambda^{-1}~~~\mathbb{D}\xi = \lambda^{-2}$$

\section{Нормальное распределение}

$$\rho_{\xi \sim N(\mu, \sigma)}(x) = \frac{1}{\sqrt{2\pi}}e^{\frac{(x-\mu)^2}{2\sigma^2}}$$

$$\mathbb{E}\xi = \mu~~~\mathbb{D}\xi = \sigma^2$$

\section{Неравенства}
\subsection{Маркова}
$$P(\xi \geq a) \leq \frac{E\xi}{a}$$

\subsection{Чебышева}
$$P(|\xi - E\xi| \geq a) \leq \frac{D\xi}{a^2}$$

\section{Простой ЗБЧ. Закон Больших Чисел}
$$P\left(\left|\frac{X_1 + \ldots + X_n}{n} - EX_1\right| \geq a\right) \leq \frac{DX_1}{na^2}$$

\section{Леммы Борелля-Кантелли}
\subsection{1-я}

Пусть есть последовательность (необязательно независимых) событий $\{A_i\}_{i=1}^{\infty}$.

Обозначим событие

$$A = \lim_{n\to \infty}\bigcap_{n=1}^{\infty}\bigcup_{i=n}^{\infty}A_i$$

Пусть ряд сходится

$$\sum_{i=1}^{\infty}P(A_i) < \infty$$

Тогда

$$P(A) = 0$$

\subsection{2-я}

Пусть есть последовательность совместно независимых событий $\{A_i\}_{i=1}^{\infty}$.

Обозначим событие 

$$A = \lim_{n\to \infty}\bigcap_{n=1}^{\infty}\bigcup_{i=n}^{\infty}A_i$$

Пусть ряд расходится

$$\sum_{i=1}^{\infty}P(A_i) \to \infty$$

Тогда

$$P(A) = 1$$

\section{Сходимости}
\subsection{Почти наверное}
$$\xi_n \overset{\textrm{п.н.}}{\to} \xi$$

Если

$$P(\{ \omega \in \Omega : \xi_n(\omega) \underset{n \to \infty}{\to} \xi(\omega) \}) = 1$$ 

или

$$P(\{ \omega \in \Omega : \xi_n(\omega) \underset{n \to \infty}{\not\to} \xi(\omega) \}) = 0$$ 

или эквивалентно

$$\forall \varepsilon > 0 : P(\{ \omega \in \Omega : \sup_{k \geq n} |\xi_k(\omega)-\xi(\omega)| > \varepsilon \}) \underset{n \to \infty}{\to} 0$$

\subsection{По вероятности}
$$\xi_n \overset{\mathbb{P}}{\to} \xi$$ 

Если

$$\forall \varepsilon > 0 : P(\{\omega \in \Omega : |\xi_n(\omega)-\xi(\omega)| > \varepsilon \}) \underset{n \to \infty}{\to} 0$$

\subsection{В p-среднем}
$$ \xi_n \overset{L^p}{\to} \xi~~~~p > 0$$

Если

$$\mathbb{E}|\xi_n(\omega)-\xi(\omega)|^p \underset{n \to \infty}{\to} 0$$

\subsection{Слабая. По распределению}
$$\xi_n \overset{d}{\to} \xi$$

Если $\forall y\in\mathbb{R}$, в которых $F_{\xi}(y)$ непрерывна

$$F_{\xi_n}(y) \underset{n \to \infty}{\to} F_{\xi}(y)$$

При этом:  

Почти наверное $\Rightarrow$ По вероятности

В среднем $\Rightarrow$ По вероятности

По вероятности $\Rightarrow$ Слабая 


\section{т. Муавра-Лапласа}

$$P(\xi_n = m)  \approx \frac{1}{\sqrt{npq}} \phi(\frac{m-np}{\sqrt{npq}})$$

$$\phi(x) = \frac{1}{\sqrt{2\pi}}e^{-\frac{x^2}{2}}$$

$$P(k \leq \xi_n \leq m)  \approx \frac{1}{\sqrt{npq}} \Big(\Phi(\frac{m-np}{\sqrt{npq}}) - \Phi(\frac{k-np}{\sqrt{npq}})\Big)$$

$$\Phi(x) = \frac{1}{\sqrt{2\pi}}\int e^{-\frac{x^2}{2}}dx$$


%%\section{ЦПТ}
%%$$\frac{S_n - \mathbb{E}S_n}{\sqrt{DS_n}} \to N(0,1)$$

\end{document}


