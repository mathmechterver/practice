\documentclass[a4paper, 14pt]{extarticle}

%% Language and font encodings
\usepackage[english, russian]{babel}
\usepackage[utf8]{inputenc}

\usepackage[a4paper,top=2cm,bottom=2cm,left=2cm,right=1.5cm,margin=15mm, lmargin=30mm]{geometry}

\title{practice2old}
\author{samstikhin}
\date{September 2018}

\begin{document}

\section*{Полная вероятность. Формула Байеса}
\subsection*{Базовый}
\begin{enumerate}
\item В первой урне 3 белых и 5 черных шаров. Во второй урне 5 белых и 5 черных шаров. Из первой урны 2 шара переложили во вторую, а затем из второй взяли шар. Какова вероятность, что он белый?
\item На склад поступило 2 партии изделий: первая – 4000 штук, вторая – 6000 штук. Средний процент нестандартных изделий в первой партии составляет 20\%, а во второй – 10\%. Наудачу взятое со склада изделие оказалось стандартным. Найти вероятность того, что оно: а) из первой партии, б) из второй партии.
\item Из 30 стрелков 12 попадает в цель с вероятностью 0,6, 8 - с вероятностью 0,5 и 10 – с вероятностью 0,7. Наудачу выбранный стрелок произвел выстрел, поразив цель. К какой из групп вероятнее всего принадлежал этот стрелок? 
\item Среди 16 тенниситов одинаковой силы есть 2 брата. Какова вероятность того, что они сыграли вместе в финале?
\item Даны натуральные числа m и n, причем m<n. Из чисел 1,2,…,n последовательно выбирают наугад
два различных числа. Найдите вероятность того, что разность между первым выбранным числом и вторым будет не меньше m.
\item Дано натуральное число n<52. Из тщательно перемешанной колоды в 52 карты одновременно были взяты n карт. На одну из этих n карт посмотрели, она оказалась тузом. После этого она возвращается в набор взятых карт и эти n карт перемешиваются. После этого из них выбирается одна карта и открывается. Найдите вероятность того, что открытая карта является тузом.
\item (Парадокс Монти Холла)Представьте, что вы стали участником игры, в которой вам нужно выбрать одну из трёх дверей. За одной из дверей находится автомобиль, за двумя другими дверями — козы. Вы выбираете одну из дверей, например, номер 1, после этого ведущий, который знает, где находится автомобиль, а где — козы, открывает одну из оставшихся дверей, например, номер 3, за которой находится коза. После этого он спрашивает вас — не желаете ли вы изменить свой выбор и выбрать дверь номер 2? Увеличатся ли ваши шансы выиграть автомобиль, если вы примете предложение ведущего и измените свой выбор?


\end{enumerate}

\newpage

\section*{Полная вероятность. Формула Байеса}
\subsection*{Домашка}
\begin{enumerate}
\item (0.5б)Показать, что
	$P(A|B) =P(A|BC)P(C|B)+P(A|B\overline{C})P(\overline{C}|B).$

	\item (0.5б)Вероятности того, что во время работы кластера из трех серверов сбой произойдет в соответственно в 1-м, 2-м и 3-м сервере, относятся как 3:2:5. Вероятности обнаружения
	1-м, 2-м и 3-м сервере соответственно равны 0,8; 0,9;
	0,9. Найти вероятность того, что возникший в кластере
	сбой будет обнаружен.
	\item (1б)Найти вероятность того, что станок, работающий в 
	момент $t_0$, не остановится до момента $t_0+t$, если известно, что
	эта вероятность зависит только от величины промежутка времени
	$(t_0,t_0+t)$ и 
	 вероятность того, что станок остановится за промежуток времени
	$\Delta t$, пропорциональна $\Delta t$ с точностью до бесконечно малых высших
	порядков относительно $\Delta t$.	
	
	\item (0.5б)В стране Зурумбии разрешено пользоваться лишь тремя мессенджерами. И за деньги. Посылка каждого сообщения -- 10 зурумбийских долларов. Один мессенджер никогда не работает, второй всегда работает, а третий передает сообщения с вероятностью 50\%. Какой именно мессенджер как работает неизвестно. У представителя РФ  в Зурумбии Камиля всего 30 зурумбийских долларов. Камиль протестировал некоторый мессенджер и он не работал, он два раза протестировал еще один и  оба раза передал сообщения. С какой вероятностью непротестированный мессенджер  всегда передает сообщения?
	
	\item (0.5б)Известно, что 96\% выпускаемой продукции соответствует стандарту. Упрощенная схема контроля признает годным с вероятностью
	0,98 каждый стандартный экземпляр аппаратуры и с 
	0,05 – каждый нестандартный экземпляр аппаратуры. Найдите вероятность того, что изделие, прошедшее контроль, соответствует стандарту.


	\item (0.5б)Имеются две урны. В одной из них находится один белый шар, в
	другой – один черный шар (других шаров урны не содержат). Выбирается наугад одна урна. В нее добавляется один белый шар и после перемешивания один из шаров извлекается. Извлеченный шар оказался
	белым. Определите вероятность того, что выбранной
	оказалась урна, которая первоначально содержала белый шар.
	
	\item (0.5)Есть 4 кубика. На трех из них окрашена белым половина граней, а на четвертом кубике всего одна грань из шести белая. Наудачу выбранный кубик подбрасывается семь раз. Найти вероятность того, что был выбран четвертый кубик, если при семи подбрасываниях белая грань выпала ровно один раз. 
\end{enumerate}
\end{document}

