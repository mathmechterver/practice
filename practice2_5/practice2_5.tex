\documentclass[a4paper, 14pt]{extarticle}

%% Language and font encodings
\usepackage[english, russian]{babel}
\usepackage[utf8]{inputenc}

\usepackage[a4paper,top=1cm,bottom=1cm,left=1cm,right=1.5cm,margin=10mm, lmargin=15mm]{geometry}

\usepackage{amsfonts,amssymb,amsmath}
\usepackage{nopageno, comment}
\usepackage{cmap}
\usepackage{ifthen}
\usepackage{indentfirst}
\usepackage{float}
\usepackage{tikz}
\usepackage{wrapfig}

\title{practice2_5}
\author{Sam Stikhin}
\date{February 2019}

\begin{document}

\section*{2.5 Слабая сходимость. ЗБЧ. ЦПТ.}
\subsection*{Пререквизиты}
\subsubsection*{Сходимость. Слабая. По распределению.}
$\xi_n \overset{d}{\to} \xi$, если для любой непрерывной функции $\phi(x)$ определенной на $\mathbb{R}$   
$$\int_{\Omega}\phi(\xi_n(\omega))\mathbb{P}(d\omega) \to \int_{\Omega}\phi(\xi(\omega))\mathbb{P}(d\omega)$$

или

$$\int_{\mathbb{R}}\phi(x)\mu_n(dx) \to \int_{\mathbb{R}}\phi(x)\mu_n(dx)$$

или

$$\forall x F_{\xi_n}(x) \to F_{\xi}(x) ~~~~ F_{\mu}(x) = \mu((-\infty, x))$$


\subsubsection*{Сходимость. В среднем.}
$$\mathbb{E}|\xi_n - \xi|^p \to 0$$

\subsubsection*{Характеристические функции}
$$\chi_\xi(t) = \mathbb{E}(e^{it\xi})$$

\subsubsection*{Закон больших чисел Чебышева}
Для \textbf{попарно независимых и одинаковораспределенных} 
$\{\xi_i\}_{i=1}^{\infty}$ с конечными \textbf{вторыми} моментами
$$\frac{\sum_{i=1}^n\xi_i}{n} \overset{\mathbb{P}}{\to} \mathbb{E}\xi_1$$	

\subsubsection*{Закон больших чисел Хинчина}
Для \textbf{в совокупности независимых и одинаковораспределенных} 
$\{\xi_i\}_{i=1}^{\infty}$ с конечными \textbf{первыми} моментами
$$\frac{\sum_{i=1}^n\xi_i}{n} \overset{\mathbb{P}}{\to} \mathbb{E}\xi_1$$	

\subsubsection*{Закон больших чисел Маркова}
Для \textbf{необязательно независимых и необязательно одинаковораспределенных} 
$\{\xi_i\}_{i=1}^{\infty}$, для которых $\sum_{i=1}^{n}\mathbb{D}(\xi_i) = o(n^2)$
$$\frac{\sum_{i=1}^n\xi_i}{n} \overset{\mathbb{P}}{\to} \mathbb{E}\xi_1$$

\subsubsection*{Усиленный Закон больших чисел Колмогорова}
Для тех же условий, что и обычные ЗБЧ
$$\frac{\sum_{i=1}^n\xi_i}{n} \overset{\textrm{п.н.}}{\to} \mathbb{E}\xi_1$$	

\subsubsection*{Классическая ЦПТ}
Для \textbf{в совокупности независимых и одинаковораспределенных} 
$\{\xi_i\}_{i=1}^{\infty}$ с конечными \textbf{вторыми} моментами удовлетворяет
$$\frac{(\sum_{i=1}^n\xi_i) - n\mathbb{E}(\xi_1)}{\sqrt{n\mathbb{D}(\xi_1)}} \overset{\mathbb{d}}{\to} N(0,1)$$	

\subsubsection*{ЦПТ Линдеберга}
Для независимых $\{\xi_i\}_{i=1}^{\infty}$ с конечными \textbf{вторыми} моментами и 
условия Линдерберга $$\forall \varepsilon > 0 \lim_{n\to \infty} \sum_{i=1}^{n}\mathbb{E}\Big[\frac{(\xi_i - \mu_i)^2}{s_n^2}\textbf{1}_{\{|X_i-\mu_i| > \varepsilon s_n\}}\Big]$$
$$\frac{(\sum_{i=1}^n\xi_i) - n\mathbb{E}(\xi_1)}{\sqrt{n\mathbb{D}(\xi_1)}} \overset{\mathbb{d}}{\to} N(0,1)$$	

	  
\newpage
\section*{2.5 Слабая сходимость. ЗБЧ. ЦПТ. Х-ские функции}
\subsection*{Практика}
\begin{enumerate}

\item Пусть $\xi_n \overset{d}{\to} \xi$ и $g(x)$ - непрерывная функция. Докажите, что $g(\xi_n) \overset{d}{\to} g(\xi)$

\item Если $\xi_n \overset{d}{\to} const$, то $\xi_n \overset{\mathbb{P}}{\to} const$

\item Пусть $\{\xi_i\}_{i=1}^{\infty}$ - положительно определенные и $E|\xi_n - \xi| \to 0$. 
Сходится ли $E|\xi_n + k| \to E|\xi+k|$ для k из $\mathbb{R_{+}}$. 
Докажите, что в общем случае это не верно.

\item Найти характеристические функции для кубика и экспоненциального распределения.

\item (1)Пусть $\xi_n$ принимает значения $n$, $-n$, 0 c вероятностями:
$\frac{1}{2\sqrt{n}}$, $\frac{1}{2\sqrt{n}}$, $1 - \frac{1}{\sqrt{n}}$
Выполнен ли закон больших чисел?

\item Пусть $\xi_1 \sim N(0,1)$. Положим $\chi_n^2 = \sum_{i=1}^{n}\xi_i^2$ и $\tau_n = \frac{\xi_{n+1}}{\sqrt{\frac{\chi_n^2}{n}}}$
Найти предельное распределение величины $\tau$

\end{enumerate}
\newpage
\subsection*{Домашка}
\begin{enumerate}

\item (1)Пусть $\xi_n \overset{d}{\to} \xi$ и $\mu_n \overset{d}{\to} \mu$. Доказать, тчо
$ \xi_n + \mu_n \overset{\textrm{d}}{\to} \xi + \mu$

\item (1)Пусть $\xi_n \overset{d}{\to} \xi$ и $\mu_n \overset{d}{\to} \mu$. Для любой непрерывной функции f.
$f(\xi_n,\mu_n) \overset{\textrm{d}}{\to} f(\xi, \mu)$

\item (2)Пусть $\alpha > 0$ и $E|\xi_n|^\alpha < \infty$ при всех n. Доказать, что следующие
утверждения эквивалентны:
\begin{enumerate}
\item $\xi_n \to \xi$ по вероятности и $E|\xi_n|^\alpha \to E|\xi|^\alpha < \infty$ при $n \to \infty$;
\item $E|\xi_n - \xi|^\alpha \to 0$ при $n \to \infty$. 
\end{enumerate}

\item (1)Вычислить характеристическую функцию распределения Лапласа.

\item (1)Пусть $\xi_n$ принимает значения $n^3$, $-n^3$, 0 c вероятностями:
\begin{enumerate}
\item $\frac{1}{2\sqrt{n}}$, $\frac{1}{2\sqrt{n}}$, $1 - \frac{1}{\sqrt{n}}$
\item $\frac{1}{2n}$, $\frac{1}{2}$, $1 - \frac{1}{n}$
\end{enumerate} 
Выполнен ли закон больших чисел?

\item (2)Пусть $\{\xi_n\}_{n=1}^{\infty}$ - последовательность независимых одинаково распределенных (н.о.р) случайных величин
с конечной дисперсией. Доказать:
$$\frac{\max{(\xi_1, \ldots, \xi_n)}}{\sqrt{n}} \overset{\textrm{d}}{\to} 0$$

%\item (2)Пусть $\{\xi_i\}_{i=1}^{\infty}$ с конечными матожиданием $\mu$ и дисперсией $\sigma^2$. 
%Положим $\eta_n = \frac{\xi_1 + \ldots + \xi_n}{\xi_1^2 + \ldots + \xi_n^2}$. Доказать, что $\eta_n$ сходится по вероятности и найти этот предел.


%%\item Пусть случайные величины $\xi_1$, $\xi_2$, ... независимы в совокупности и $\xi_n \to 0$ п. н. 
%%Доказать, что ряд $\sum\limits_{n=1}^{+\infty} \mathbb{P}\lbrace{\left|\xi_n\right| > 1\rbrace}$ сходится.

%%\item Пусть $\xi_n$ - последовательность независимых и равномерно распределенных на [0,1] случайных величин. 
%%Найдите распределение случайной величины $m_n=\min (\xi_1,\ldots,\xi_n)$. Докажите, что $m_n$ стремится почти наверное к 0

%%\item (1) Пусть дана последовательность независимых одинаково распределенных случайных величин $X_1$, ..., $X_n$. Правило $\hat{\theta}(X_1, ..., X_n)$ называется состоятельной оценкой параметра $\theta$ распределения, если $\hat{\theta}_n \overset{\mathbb{P}}{\to} \theta$. Докажите, что если $E_{|X_1|^2}$ существует, то выборочное среднее $\bar{X} = \frac{1}{n}(X_1 + ... + X_n)$ является состоятельной оценкой математического ожидания $E_{X_1}$.

%%\item (1) Пусть $\xi_1$, $\xi_2$, ... - последовательность независимых случайных величин. $E \xi_n^2 < \sigma^2$ и $E\xi_n = 0$. Докажите, что ряд $\sum\limits_{n=1}^{+\infty}\frac{\xi_n}{2^n}$ сходится с вероятностью 1.

%%\item (1) Пусть дана последовательность независимых случайных величин $X_n$. Докажите, что если $X_n$ сходится почти всюду, то предел - константа.

%%\item  (2) Пусть дана последовательность независимых одинаково распределенных случайных величин $X_1,\dots,X_n$.
	  %%Определим эмпирическую функцию распределения:
	  %%$$F^*(y)=\frac{1}{n}\sum_{i=1}^n 1_{\{\omega\,|\,X_i(\omega)\leq y\}}\qquad \forall y\in\mathbb{R}.$$
	  %%Докажите, что эмпирическая функция распределения является состоятельной оценкой функции распределения $F$ (для всех $y$)

%%\item (1) Пусть $\xi_n$ принимает значения $n$, $-n$ и $0$ с вероятностями $\frac{1}{2n^2}$, $\frac{1}{2n^2}$ и $1 - \frac{1}{n^2}$.
%%Выполнен ли для этой последовательности закон больших чисел? 

%%\item Пусть $E|\xi_n| \to E|\xi|$. Сходится ли $E|\xi_n + k| \to E|\xi+k|$?

%%\item Пусть $\alpha > 0$ и $E|\xi_n|^\alpha < \infty$ при всех n. Доказать, что следующие
%%утверждения эквивалентны:
%%1) $\xi_n \to \xi$ по вероятности и $E|\xi_n|^\alpha \to E|\xi|^\alpha < \infty$ при $n \to \infty$;
%%2) $E|\xi_n - \xi|^\alpha \to 0$ при $n \to \infty$. 

%\item (1) Пусть $\xi_1$, $\xi_2$, ... - последовательность независимых случайных величин. $E \xi_n^2 < \sigma^2$ и $E\xi_n = 0$. Докажите, что ряд $\sum\limits_{n=1}^{+\infty}\frac{\xi_n}{2^n}$ сходится с вероятностью 1.


%\item Пусть случайные величины $\xi_1$, $\xi_2$, ... независимы в совокупности и $\xi_n \to 0$ п. н. 
%Доказать, что ряд $\sum\limits_{n=1}^{+\infty} \mathbb{P}\lbrace{\left|\xi_n\right| > 1\rbrace}$ сходится.

%\item Пусть $\xi_n$ - последовательность независимых и равномерно распределенных на [0,1] случайных величин. 
%Найдите распределение случайной величины $m_n=\min (\xi_1,\ldots,\xi_n)$. Докажите, что $m_n$ стремится почти наверное к 0

%\item (1) Пусть дана последовательность независимых одинаково распределенных случайных величин $X_1$, ..., $X_n$. Правило $\hat{\theta}(X_1, ..., X_n)$ называется состоятельной оценкой параметра $\theta$ распределения, если $\hat{\theta}_n \overset{\mathbb{P}}{\to} \theta$. Докажите, что если $E_{|X_1|^2}$ существует, то выборочное среднее $\bar{X} = \frac{1}{n}(X_1 + ... + X_n)$ является состоятельной оценкой математического ожидания $E_{X_1}$.

%\item (1) Пусть дана последовательность независимых случайных величин $X_n$. Докажите, что если $X_n$ сходится почти всюду, то предел - константа.

%\item  (2) Пусть дана последовательность независимых одинаково распределенных случайных величин $X_1,\dots,X_n$.
%	  Определим эмпирическую функцию распределения:
%	  $$F^*(y)=\frac{1}{n}\sum_{i=1}^n 1_{\{\omega\,|\,X_i(\omega)\leq y\}}\qquad \forall y\in\mathbb{R}.$$
%	  Докажите, что эмпирическая функция распределения является состоятельной оценкой функции распределения $F$ (для всех $y$)
\end{enumerate}

\end{document}
