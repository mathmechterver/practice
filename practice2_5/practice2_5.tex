\documentclass[a4paper, 14pt]{extarticle}

%% Language and font encodings
\usepackage[english, russian]{babel}
\usepackage[utf8]{inputenc}

\usepackage[a4paper,top=1cm,bottom=1cm,left=1cm,right=1.5cm,margin=10mm, lmargin=15mm]{geometry}

\usepackage{amsfonts,amssymb,amsmath}
\usepackage{nopageno, comment}
\usepackage{cmap}
\usepackage{ifthen}
\usepackage{indentfirst}
\usepackage{float}
\usepackage{tikz}
\usepackage{wrapfig}

\title{practice2_2}
\author{Sam Stikhin}
\date{February 2019}

\begin{document}

\section*{2.3 Сходимости случайных величин.}
\subsection*{Пререквизиты}
\textbf{Леммы Борелля-Кантелли}
\textbf{1-я}

Пусть есть последовательность (необязательно независимых) событий $\{A_i\}_{i=1}^{\infty}$

Обозначим событие $A = \lim_{n\to \infty}\bigcap_{n=1}^{\infty}\bigcup_{i=n}^{\infty}A_i$

Пусть ряд сходится $\sum_{i=1}^{\infty}P(A_i) < \infty$

Тогда $P(A) = 0$

\textbf{2-я}

Пусть есть последовательность совместно независимых событий $\{A_i\}_{i=1}^{\infty}$

Обозначим событие $A = \lim_{n\to \infty}\bigcap_{n=1}^{\infty}\bigcup_{i=n}^{\infty}A_i$

Пусть ряд расходится $\sum_{i=1}^{\infty}P(A_i) \to \infty$

Тогда $P(A) = 1$

\textbf{Почти наверное}
$$\xi_n \overset{\textrm{п.н.}}{\to} \xi$$

Если

$$P(\{ \omega \in \Omega : \xi_n(\omega) \underset{n \to \infty}{\to} \xi(\omega) \}) = 1$$ 

или

$$P(\{ \omega \in \Omega : \xi_n(\omega) \underset{n \to \infty}{\not\to} \xi(\omega) \}) = 0$$ 

или эквивалентно

$$\forall \varepsilon > 0 : P(\{ \omega \in \Omega : \sup_{k \geq n} |\xi_k(\omega)-\xi(\omega)| > \varepsilon \}) \underset{n \to \infty}{\to} 0$$

\textbf{По вероятности}
$$\xi_n \overset{\mathbb{P}}{\to} \xi$$ 

Если

$$\forall \varepsilon > 0 : P(\{\omega \in \Omega : |\xi_n(\omega)-\xi(\omega)| > \varepsilon \}) \underset{n \to \infty}{\to} 0$$

\textbf{Закон больших чисел}
Говорят, что последовательность случайных величин $\{\xi_i\}_{i=1}^{\infty}$ с конечными первыми моментами удовлетворяет закону больших чисел (ЗБЧ), если
$$\frac{\sum_{i=1}^n\xi_i}{n} - \frac{\sum_{i=1}^n\mathbb{E}\xi_i}{n} \overset{\mathbb{P}}{\to} 0$$
	
	  
\newpage
\section*{2.3 Сходимости случайных величин.}
\subsection*{Практика}
\begin{enumerate}

\item Точка путешествует по целым числам. Каждый раз она шагает на $1$, с вероятностью $p$ — вправо, с вероятностью $1-p$ — влево. 
Докажите, что при $p \neq \frac{1}{2}$ вероятность того, что она вернется в исходное положение бесконечное число раз равна $0$.

\item Пусть $\xi_n \to \xi$ п.н и $g(x)$ - непрерывная функция. Докажите, что $g(\xi_n) \to g(\xi)$ п.н.

\item Пусть $(\xi_n - \xi)^2 \to 0$ п.н. Доказать, что $\xi_n \to \xi$ п.н.

\item Пусть $\xi_n \overset{\mathbb{P}}{\to} \xi$ и $g(x)$ - непрерывная, дифференцируемая и монотонно возрастающая функция. Докажите, что $g(\xi_n) \overset{\mathbb{P}}{\to} g(\xi)$.

\item Пусть $\xi_n$ принимает значения $2^n$, $- 2^n$ и $0$ с вероятностями $2^{-(2n+1)}$, $2^{-(2n+1)}$ и $1 - 2^{-2n}$ соответственно. 
Пусть Докажите, что $\xi_n \overset{\mathbb{P}}{\to} 0$.


\end{enumerate}
\newpage
\subsection*{Домашка}
\begin{enumerate}
\item (2) Точка путешествует по целым числам. Каждый раз она шагает на $1$, с вероятностью $p$ — вправо, с вероятностью $1-p$ — влево. 
Докажите, что при $p = \frac{1}{2}$ вероятность того, что она вернется в исходное положение бесконечное число раз равна $1$.

\item Пусть $\xi_n \overset{\textrm{п.н.}}{\to} 1$ и $\nu_n \overset{\textrm{п.н.}}{\to} 1$. 
Тогда 
\begin{enumerate}
\item $\xi_n + \nu_n \overset{\textrm{п.н.}}{\to} 2$
\item $\xi_n\nu_n \overset{\textrm{п.н.}}{\to} 1$
\end{enumerate}


\item Пусть случайные величины $\xi_1$, $\xi_2$, ... независимы в совокупности и $\xi_n \to 0$ п. н. 
Доказать, что ряд $\sum\limits_{n=1}^{+\infty} \mathbb{P}\lbrace{\left|\xi_n\right| > 1\rbrace}$ сходится.

\item Пусть $\xi_n$ - последовательность независимых и равномерно распределенных на [0,1] случайных величин. 
Найдите распределение случайной величины $m_n=\min (\xi_1,\ldots,\xi_n)$. Докажите, что $m_n$ стремится почти наверное к 0

\item (2) Точка путешествует по целочисленной решетке в $\mathbb{R}^3$. Каждый раз она шагает на расстояние 1, все 6 возможных направлений шага равновероятны. Докажите, что с вероятностью 1 она вернется в исходное положение не более чем конечное число раз.

\item (1) Пусть дана последовательность независимых одинаково распределенных случайных величин $X_1$, ..., $X_n$. Правило $\hat{\theta}(X_1, ..., X_n)$ называется состоятельной оценкой параметра $\theta$ распределения, если $\hat{\theta}_n \overset{\mathbb{P}}{\to} \theta$. Докажите, что если $E_{|X_1|^2}$ существует, то выборочное среднее $\bar{X} = \frac{1}{n}(X_1 + ... + X_n)$ является состоятельной оценкой математического ожидания $E_{X_1}$.

\item (1) Пусть $\xi_1$, $\xi_2$, ... - последовательность независимых случайных величин. $E \xi_n^2 < \sigma^2$ и $E\xi_n = 0$. Докажите, что ряд $\sum\limits_{n=1}^{+\infty}\frac{\xi_n}{2^n}$ сходится с вероятностью 1.

\item (1) Пусть дана последовательность независимых случайных величин $X_n$. Докажите, что если $X_n$ сходится почти всюду, то предел - константа.

\item  (2) Пусть дана последовательность независимых одинаково распределенных случайных величин $X_1,\dots,X_n$.
	  Определим эмпирическую функцию распределения:
	  $$F^*(y)=\frac{1}{n}\sum_{i=1}^n 1_{\{\omega\,|\,X_i(\omega)\leq y\}}\qquad \forall y\in\mathbb{R}.$$
	  Докажите, что эмпирическая функция распределения является состоятельной оценкой функции распределения $F$ (для всех $y$)

\item (1) Пусть $\xi_n$ принимает значения $n$, $-n$ и $0$ с вероятностями $\frac{1}{2n^2}$, $\frac{1}{2n^2}$ и $1 - \frac{1}{n^2}$.
Выполнен ли для этой последовательности закон больших чисел? 

\item Пусть $E|\xi_n| \to E|\xi|$. Сходится ли $E|\xi_n + k| \to E|\xi+k|$?

\item Пусть $\alpha > 0$ и $E|\xi_n|^\alpha < \infty$ при всех n. Доказать, что следующие
утверждения эквивалентны:
1) $\xi_n \to \xi$ по вероятности и $E|\xi_n|^\alpha \to E|\xi|^\alpha < \infty$ при $n \to \infty$;
2) $E|\xi_n - \xi|^\alpha \to 0$ при $n \to \infty$. 

%\item (1) Пусть $\xi_1$, $\xi_2$, ... - последовательность независимых случайных величин. $E \xi_n^2 < \sigma^2$ и $E\xi_n = 0$. Докажите, что ряд $\sum\limits_{n=1}^{+\infty}\frac{\xi_n}{2^n}$ сходится с вероятностью 1.


%\item Пусть случайные величины $\xi_1$, $\xi_2$, ... независимы в совокупности и $\xi_n \to 0$ п. н. 
%Доказать, что ряд $\sum\limits_{n=1}^{+\infty} \mathbb{P}\lbrace{\left|\xi_n\right| > 1\rbrace}$ сходится.

%\item Пусть $\xi_n$ - последовательность независимых и равномерно распределенных на [0,1] случайных величин. 
%Найдите распределение случайной величины $m_n=\min (\xi_1,\ldots,\xi_n)$. Докажите, что $m_n$ стремится почти наверное к 0

%\item (1) Пусть дана последовательность независимых одинаково распределенных случайных величин $X_1$, ..., $X_n$. Правило $\hat{\theta}(X_1, ..., X_n)$ называется состоятельной оценкой параметра $\theta$ распределения, если $\hat{\theta}_n \overset{\mathbb{P}}{\to} \theta$. Докажите, что если $E_{|X_1|^2}$ существует, то выборочное среднее $\bar{X} = \frac{1}{n}(X_1 + ... + X_n)$ является состоятельной оценкой математического ожидания $E_{X_1}$.

%\item (1) Пусть дана последовательность независимых случайных величин $X_n$. Докажите, что если $X_n$ сходится почти всюду, то предел - константа.

%\item  (2) Пусть дана последовательность независимых одинаково распределенных случайных величин $X_1,\dots,X_n$.
%	  Определим эмпирическую функцию распределения:
%	  $$F^*(y)=\frac{1}{n}\sum_{i=1}^n 1_{\{\omega\,|\,X_i(\omega)\leq y\}}\qquad \forall y\in\mathbb{R}.$$
%	  Докажите, что эмпирическая функция распределения является состоятельной оценкой функции распределения $F$ (для всех $y$)
\end{enumerate}

\end{document}
