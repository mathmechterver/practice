\documentclass[a4paper, 14pt]{extarticle}

%% Language and font encodings
\usepackage[english, russian]{babel}
\usepackage[utf8]{inputenc}

\usepackage[a4paper,top=2cm,bottom=2cm,left=2cm,right=1.5cm,margin=15mm, lmargin=30mm]{geometry}

\title{practice10}
\author{samstikhin}
\date{September 2018}

\begin{document}

\section*{2.1 О-символика. Асимптотики. Пуассон.}
\subsection*{Практика}
Полезные формулы
$$n! = \sqrt{2\pi n}\Big(\frac{n}{e}\Big)^n + o(1)$$
$$n(1+o(1)) = n + o(n)$$
$$f(n) = o(g(n)), ~~~if \lim_{n \to \infty} \frac{f}{g} = 0$$
$$f(n) \sim g(n), ~~~if \lim_{n \to \infty} \frac{f}{g} = 1$$
$$f(n) = O(g(n)), ~~~if \lim_{n \to \infty} \frac{f}{g} = C, ~~C > 0$$
\begin{enumerate}
\item Найти предел $\lim_{n\to \infty}\frac{4n^{228} + 2007n^{26.78} + 123\sqrt[5]{n}}{3n^{228} + 2019n^{0.(3)} + \sqrt[500]{n}}$
\item Найти асимптотику $C_n^k$, $C_{2n}^n$ 
\item Доказать теорему Пуассона: 
Eсли\begin{enumerate}
    \item $\lim_{n \to \infty}p_n = 0$
    \item $\lim_{n\to \infty}np_n = \lambda$
    \item $\lambda > 0$
    \item k - константа
\end{enumerate}
$$\lim_{n \to \infty}C_n^{k}p_n^k q_n^{n-k} = \frac{\lambda^{k}e^{-\lambda}}{k!}$$
\item Среди семян пшеницы $0,6\%$ семян сорняков. Какова вероятность при случайном отборе 1000 семян обнаружить 
\begin{enumerate}
    \item 0 семян сорняков
    \item ровно 6 семян сорняков
    \item меньше 3 семян сорняков
    \item больше 2 семян сорняков
\end{enumerate}
\item (*Если успеем) Найти асимптотику $C_{n^2}^{n}$


\end{enumerate}
\newpage

\section*{2.1 О-символика. Асимптотики. Пуассон.}
\subsection*{Домашка}
\begin{enumerate}
\item (1)Докажите, что невозможно отсортировать массив чисел быстрее, чем за $O(n\log{n})$. (Как угодно, хоть стирлингом, хоть через деревья)
\item (1) Найти ассимптотику для $C_{4n}^{n}$
\item (1.5)Найти ассимптотику $P(4n+k,2n,n,n,k) = \frac{(4n+k)!}{(2n)!n!n!k!}$. k - константа!  
\item (1)Вероятность попадания в цель при одном выстреле равна 0,001.
Для поражения цели необходимо не менее двух попаданий. Произведено
5000 выстрелов. Найти вероятность поражения цели
\item (1.5)Некоторая машина состоит из 10 тысяч деталей. Каждая деталь независимо от других деталей может оказаться неисправной с вероятностью $p_i$, причём 
\begin{itemize}
    \item для $n_1 = 1000$ деталей $p_1 = 0,0003$
    \item для $n_2 = 2000$ деталей $p_2 = 0,0002$
    \item для $n_3 = 7000$ деталей $p_3 = 0,0001$
\end{itemize}Машина не
работает, если в ней неисправны хотя бы две детали. Найти вероятность
того, что машина не будет работать.
\item (2)Загадана строка над алфавитом \{'a', 'b', 'c'\} длины n, n делится на 3. За один ход разрешается спросить некоторую строку длины n. Если отгадана вся строка, то ничего не происходит. Если загаданная строка и строка-запрос совпадают в ровно $\frac{n}{3}$ позициях, то мы победили. Разрешается сделать не более $\frac{n}{3}$ запросов.
Оцените приближенно (посчитав соответствующую асимптотику), вероятность победы при использовании следующей стратегии: на каждом ходу спрашивается одна строка равновероятно из всех $3^n$ возможных.




\end{enumerate}

\end{document}


