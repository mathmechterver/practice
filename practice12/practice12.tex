\documentclass[a4paper, 14pt]{extarticle}
%% Language and font encodings
\usepackage[english, russian]{babel}
\usepackage[utf8]{inputenc}

\usepackage[a4paper,top=2cm,bottom=2cm,left=2cm,right=1.5cm,margin=15mm, lmargin=30mm]{geometry}

\title{practice12}
\author{samstikhin}
\date{September 2018}

\begin{document}

\section*{Непрерывные случайные величины. Плотность.}
\subsection*{Классная работа}
\begin{enumerate}
\item Случайная величина $\xi$ имеет функцию распределения $F_{\xi}$, определенную равенством
$$F_\xi(x) =\left\{
	\begin{array}{cc}
	4^{-x}, & x\leq 0 \\
	0, & x>0
	\end{array}\right.$$
Найдите вероятности 
$P(\xi \geq -2)$ и $P(-1 < \xi < 0)$. Найдите плотность $\rho_{\xi}(x)$.
\item Случайная величина $\xi$ имеет непрерывную функцию распределения $F_{\xi}(x)$. 
Найдите функцию распределения случайной величины $\eta=3-2\xi$.
\item Вспомним задачу о двух лыжниках. 2 лыжника условились о встрече в промежуток времени $[0,1]$.
    Момент прихода на встречу каждым выбирается равновероятно. 
    Пусть $\theta = |t_1 - t_2|$. Найдите:
    \begin{enumerate}
    \item функцию распределения $F_{\theta}(x)$
    \item плотность вероятности $\rho_{\theta}(x)$
    \item матожидание $E\theta$
    \item дисперсию $E\theta$
    \end{enumerate}
\item Функция распределения случайной величины $\xi$ имеет вид $F_{\xi}(x)=a + b\arctg(x)$. 
Найдите:
\begin{enumerate}
\item параметры $a$ и $b$
\item плотность вероятности.
\end{enumerate} 

\item Дана плотность вероятности $\rho_{\xi} = \frac{1}{\sigma \sqrt{2\pi}} e^{-\frac{(x-\mu)^2}{2\sigma^2}}$.
Доказать что:
\begin{enumerate}
\item $E_{\xi}= \mu$
\item $\sigma_{\xi}= \sigma$

\end{enumerate} 



\newpage

\end{enumerate}
\section*{Непрерывные случайные величины. Плотность.}
\subsection*{Домашняя работа}
\begin{enumerate}
\item (0.5б)В квадрате $[0,1]^2$ наугад выбирается точка $\omega=(\omega_1,\omega_2)$. Случайная величина $\xi$ задается равенством $\xi(\omega)=\omega_1+\omega_2$. Найдите функцию распределения $F_{\xi}(x)$.
	\item (0.5б)Случайная величина $\xi$ имеет непрерывную функцию распределения $F_{\xi}(x)$. 
Найдите функцию распределения случайной величины $\eta=1-3\xi^2$.
\item  (1б)Плотность распределения случайной величины $\xi$ задана формулой
	$$f_\xi(x) =\left\{
	\begin{array}{cc}
	Cx^{-3/2}, & x\geq 1 \\
	0, & x<1
	\end{array}\right.$$
	Найти: \begin{enumerate}
		\item постоянную $C$; 
		\item плотность распределения
	$\eta=1/\xi$ 
		\item $P\{0,1<\eta<0,2\}$.
	\end{enumerate}
	\item (0.75б)Случайная величина $\xi$ имеет распределение Коши с плотностью $\rho_{\xi}(x)=\frac{1}{\pi(1+x^2)}$. 
Найдите плотность и функцию распределения случайной величины $\eta = \frac{1}{1+\xi^2}$. 
	\item (0.75б)Случайная точка $A$ имеет в круге радиуса $R$
	равномерное распределение. Найти математическое
	ожидание и дисперсию расстояния  точки $A$ от центра.
	\item (2б)3 лыжника условились о встрече в промежуток времени $[0,1]$. 
	Момент прихода на встречу каждым выбирается наудачу в пределах
	указанного часа. $\theta = \max{(|t_1-t_2|,|t_2 - t_3|,|t_1-t_3|)}$
	Найдите:
    \begin{enumerate}
    \item функцию распределения $F_{\theta}(x)$
    \item плотность вероятности $\rho_{\theta}(x)$
    \item матожидание $E\theta$
    \item дисперсию $E\theta$
    \end{enumerate}
\end{enumerate}	

\end{document}


