\documentclass[a4paper, 14pt]{extarticle}

\usepackage[english, russian]{babel}
\usepackage[utf8]{inputenc}
\usepackage[a4paper,top=2cm,bottom=2cm,left=2cm,right=1.5cm,margin=15mm, lmargin=30mm]{geometry}

\begin{document}
\section{Ликбез по ДМ. Классическая вероятность.}
\subsection{Дополнительный}
\begin{enumerate}
\item (0.5б) Докажите, что среди чисел, записываемых только единицами, есть число, которое делится на 2017.

\textbf{Решение:}
Пусть $a_n$ - число, состоящее из n единиц. По принципу Дирихле среди чисел, состоящих только из единиц, найдутся 2 таких, которые дают одинаковые остатки при делении на 2017. Вычтем из большего меньшее и получим число вида $a_{r}*10^n$. Так как число 2017 - простое, $10^n$ не делится на 2017, значит $a_r$ делится на 2017.


\item (0.5б) Найти коэффициент при $x^{29}$ для $(1 + x^5 + x^7 + x^9)^{100}$.

\textbf{Решение:}
Возможные комбинации получение числа 29: 
$5+5+5+7+7 = 29$ и $9+5+5+5$. Значит при раскрытии степени нам надо взять степени $x$ из либо трех скобок с "5", двух с "7" и 95 с "0" либо четырех скобок с "5", 1 с "9" и 95 с "0". Это можно сделать с помощью полиномиального коэффициента для каждой из комбинаций: $$\frac{100!}{3!2!95!} + \frac{100!}{4!1!95!}$$.  


\item (0.5б) Доказать что:
$$\sum_{k=0}^{n} \big(C_n^k\big)^2 = C_{2n}^n$$


\textbf{Решение:}
Разделим $2n$ предметов на 2 группы по $n$. Очевидно, что чтобы выбрать n предметов, необходимо взять какое-то количество $k$ из первой группы и $n-k$ из второй для всех возможных $k$. Тогда $$C_{2n}^n = \sum_{k=0}^{n} C_n^k C_n^{n-k} = \sum_{k=0}^{n} \big(C_n^k\big)^2 $$

\addtocounter{enumi}{1}
\item (1б) Найти число циклических последовательностей длины 7 из 4 элементов.

\textbf{Решение:}
Обозначим за A - множество последовательностей, которые имеет 7 различных циклических сдвигов. Заметим, что при склейке каждой из последовтельностей получится множество размером $\frac{|A|}{7}$. Заметим, что множество A - это все последовтельности длины 7 из 4 элементов,кроме последовательностей, в которых все элементы одинаковые. Следовательно $|A| = 4^7-4$. Действительно,если какие-то два циклических сдвига последовательности равны, то последовательность имеет минимальный период отличный от 7. Так как минимальный период - число, которое делит все периоды, а 7 - период, то минимальный период может быть равен только 7 и 1. Случай с 7 мы рассмотрели, а случай с 1 - это 4 последовтельности, в которых все элементы одинаковые. Таких будет 4 последовательности. Таким образом получаем, что количество равно $\frac{4^7−4}{7} + 4$ 

\item (0.5б) Дать формулы вероятности $P_n$ того, что среди тринадцати карт, извлеченных из 52 карт, $n$ карт окажутся пиковой мастью.


\textbf{Решение:}
Количество способов взять $n$ карт пиковой масти: $C_{13}^n$. Количество способов взять оставшиеся карты $C_{52-13}^{13-n}$. Всего комбинаций взять 13 карт: $C_{52}^{13}$. Итого: $\frac{C_{13}^n C_{52-13}^{13-n}}{C_{52}^{13}}$

\item (0.5б) В программе к экзамену по теории вероятностей 75 вопросов. 
Студент знает 50 из них. В билете 3 вопроса. Найдите вероятность того, 
что студент знает хотя бы два вопроса из вытянутого им билета. 


\textbf{Решение:}
Количество способов взять билет с 3 известными вопросами: $C_{50}^3$. Количество способов взять билет с 2 известными вопросами и одним неизвестным: $C_{50}^2 C_{25}^1$. Всего комбинаций билетов: $C_{75}^3$. Итого: $$\frac{C_{50}^3 + C_{50}^2 C_{25}^1}{C_{75}^3}$$

\item (0.5б) У театральной кассы стоят в очереди $2n$ человек. Среди
	них $n$ человек имеют лишь банкноты по 1000 рублей, а остальные —
	только банкноты по 500 рублей. Билет стоит 500 рублей. Каждый покупатель
	приобретает по одному билету. В начальный момент в кассе нет денег.
	Чему равна вероятность того, что никто не будет ждать сдачу?

\textbf{Решение:}
Количество способов распределить n купюр по 500 рублей по 2n людям - это $C_{2n}^n$. Посчитаем количество способов распределить купюры по 500 рублей так, чтобы на каждом префиксе купюр по 500 рублей было бы не меньше, чем по 1000. Эта задача эквивалентна задаче о количестве правильных скобочных последовательностей (левая скобка - посетители с 500, правая - с 1000). Как известно,количество ПСП длины n-это n-ое число Каталана $K_n = \frac{1}{n+1}C_{2n}^n$. Получаем,чтовероятность равна $\frac{1}{n+1}$

\item (1б) Как-то раз 3 ковбоя Хороший, Плохой и Злой не поделили девушку низкой социальной ответственности в одном кабаре и решили устроить дуэль. Хороший попадает в цель с вероятностью $p_1$, Плохой с вероятностью $p_2$, а Злой с вероятностью $p_3$. Каждый выбирает цель с наибольшей меткостью из оставшихся (суицид никто совершать не будет). Сначала стреляет Хороший, потом Плохой, потом Злой и потом опять Хороший и.т.д. Какова вероятность Хорошему парню выйграть дуэль и заполучить девушку.


\textbf{Решение:}
Рассмотрим только случай ($p_1$>$p_2$>$p_3$), остальные аналогичные и чисто технические.
\newline
$q_i = 1-p_i$
Есть 2 варианта: либо Хороший попал, либо мы прошли круг и все промазали и мы вернулись в исходную точку.
$$P_X = P_{branch} + q_1q_2q_3P_X$$
$P_{branch}$ - вероятность того, что Хороший попал и начинается перестрелка со Злым.
$$P_{branch} = (p_1 + (1-p_3)) + p_1 + q_1q_3p_1 + (q_1q_3)^2p_1 + \ldots = (p_1 + (1-p_3)) + \frac{p_1}{1-q_1q_3}$$
Итог:
$$P_X = \frac{P_{branch}}{1-q_1q_2q_3} = \frac{(p_1 + (1-p_3))(1-q_1q_3) + p_1}{(1-q_1q_3)(1-q_1q_2q_3)}$$


\item (0.5б)В ящике находится $a$ белых и $b$ черных шаров. Если достают черный шар, то его возвращают и кладут еще один черный шар. Если вытащен белый шар, то процесс прекращается. Определить вероятности того, что белый шар вытащили на четном и нечетном шаге.


\textbf{Решение:}
Нам нужно взять белый шар на нечетном шаге, значит на всех предыдущих шагах мы брали черные шары. Взять белый шар на $2k-1$ шаге равна:
$$P(2k-1) = \Big(\frac{a}{a+b+2k}\Big)*
\Big(\frac{b}{a+b}\Big)*
\Big(\frac{(b+2k-1)!(a+b)!}{b!(a+b+2k-1)!}\Big)$$
Первая скобка - взятие белого шара, третья и вторая - взятие черных шаров. Итого нужно посчитать вот такую сумму.
$$\sum_{k=1}^{\infty}P(2k-1)$$

Как это сокращать, я не знаю. Накину 3б тому, кто научится это делать.


Для четных строится аналогичное страшное условие:)

По сути нам не нужно доказывать сходимость, так как мы знаем, что это вероятность и она меньше 1, так как правил мы не нарушали. Но для особо занудных: рассмотрите эксперимент, в котором новые шары не добавляются. В этом случае вероятность взять белый шар не уменьшается, значит общая сумма будет строго больше рассматриваемой. 

$$\sum_{k=1}^{\infty}\frac{a}{(a+b)}*(\frac{b}{(a+b)})^{2k-1} = \frac{ab}{(a+b)^2}*\frac{1}{1-(\frac{b}{a+b})^2} = \frac{ab}{a^2 + 2ab} = \frac{b}{a + 2b}$$
\end{enumerate}

\end{document}

