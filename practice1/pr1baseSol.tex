\documentclass[a4paper, 14pt]{extarticle}

\usepackage[english, russian]{babel}
\usepackage[utf8]{inputenc}
\usepackage[a4paper,top=2cm,bottom=2cm,left=2cm,right=1.5cm,margin=15mm, lmargin=30mm]{geometry}

\title{solutionBase1}
\author{Sam Stikhin}
\date{September 2018}

\begin{document}
\begin{enumerate}
\section*{Ликбез по ДМ. Классическая вероятность.}
\subsection*{Базовый}
\item В ящике лежит 100 флажков: красных, зелёных,
жёлтых, синих. Какое наименьшее
число флажков надо взять не глядя, чтобы
среди них нашлось не менее 10 одноцветных?


\textbf{Решение:}
Если мы достанем 37 флажков, то по принципу Дирихле, флажка хотя бы одного цвета будет точно 10. ($9*4=36<37$) 
\newline
\textbf{Ответ:} 37

\item Найти коэффициент при $x^{16}$ для $(1 + x^2)^{10}$.


\textbf{Решение:}
Замена $y=x^2$. Теперь нам надо найти коэффициент при $y^8$. Раскрыв перемножение получим 10 скобок: $(1+y)\ldots(1+y)$. Чтобы получить $y^8$ нужно взять "$y$" из 8 скобок и "1" из 2х оставшихся. Это в точности число способов выбрать 8 элементов из 10: $C_{10}^8$.
\newline
\textbf{Ответ:}$C_{10}^8$


\item Доказать формулу: $C_{n}^k = C_{n-1}^k + C_{n-1}^{k-1}$

\textbf{Решение:}
$C_{n-1}^k + C_{n-1}^{k-1} = \frac{(n-1)!}{(k)!(n-1-k)!} + \frac{(n-1)!}{(k-1)!(n-k)!} = \frac{(n-1)!(n-k)}{(k)!(n-k)!} + \frac{(n-1)!k}{k!(n-k)!} = \frac{(n-1)!(n - k + k)}{k!(n-k)!} = \frac{n!}{(k)!(n-k)!} = C_n^k$


\item Доказать формулу сочетаний с повторениями: 
$$\overline{C_n^k} = C_{n+k-1}^k$$

\textbf{Решение:} Задача сводится к тому, у нас есть $k$ шариков и $n$ цветов и нужно посчитать сколько способов назначить шарику цветов. Выстроим шарики в ряд и будем вставлять между ними перегородки. Если между шариками нет перегородки, то красим их в один цвет. Всего перегородок будет $n-1$ штука, так как цветов $n$ (у краевых шаров перегородка только с одной стороны). Если 2 перегородки стоят рядом - то шариков какого-то цвета нет. Итого получили $k$ шариков и $n-1$ перегородок или же $k+n-1$ мест в ряде, для которых нужно выбрать куда поставить перегородки, ведь шарики после этого поставятся единственным способом. Это и будет чисто сочетаний $C_{n+k-1}^{k}$


\item Сколькими способами можно составить хоровод из n девушек?


\textbf{Решение:}
Заметим, что можем расцепить хоровод в $n$ местах и получить $n$ разных последовательностей девушек длины $n$(все циклические сдвиги). Значит хороводов в $n$ раз меньше, чем последовательностей длины $n$. Итого: $\frac{n!}{n}$ \newline
\textbf{Ответ:}$(n-1)!$


\item Монета брошена 2 раза (3 раза, n раз). Найти вероятность того,
	что хотя бы один раз появится орел.
	

\textbf{Решение:}
Пойдем от противного и найдем вероятность того, что ни один орел не появился: значит были только решки. Вероятность этого $\frac{1}{2^n}$. Тогда искомая вероятность $1 - \frac{1}{2^n}$ \newline
\textbf{Ответ:}$1 - \frac{1}{2^n}$


\item Брошены две игральные кости. Найти вероятность, 
что сумма выпавших очков равна 7 (9 очков, 12 очков).


\textbf{Решение:}
Всего вариантов выпадение 2х костей: 36. Хороших вариантов у нас 6: (1,6), (6,1), (2,5), (5,2), (3,4), (4,3). Итого $\frac{1}{6}$ для 7. Для 9: $\frac{1}{9}$. Для 12: $\frac{1}{36}$ 



\item  На полке в случайном порядке расставлено
40 книг, среди которых находится трехтомник А. С. 
Пушкина. Найти вероятность того, что эти тома стоят в 
порядке возрастания слева направо (но не обязательно рядом).


\textbf{Решение:}
Вариантов расставить книги: $40!$. Вариантов расставить 3 томика Пушкина: $C_{40}^3$, так как порядок мы точно знаем. Вариантов расставить все книги кроме томиков Пушкина: $37!$. Ответ: $\frac{C_{40}^3 37!}{40!}$

\item Партия продукции состоит из десяти изделий, среди которых два
изделия дефектные. Какова вероятность того, что из пяти отобранных
наугад и проверенных изделий:\begin{enumerate}
	\item ровно одно изделие дефектное;
	\item ровно два изделия дефектные;
	\item хотя бы одно изделие дефектное?
\end{enumerate}


\textbf{Решение:}
(a)Выбрать хорошие делати: $C_{10}^4$, выбрать плохую деталь: $C_{2}^1$. Всего способов выбрать 5 деталей: $C_{10}^5$. Итого: $\frac{C_{10}^4C_{2}^1}{C_{10}^5}$. \newline 
(b) $\frac{C_{10}^3}{C_{10}^5}$
\newline
(c) $\frac{C_{10}^4C_{2}^1}{C_{10}^5} + \frac{C_{10}^3}{C_{10}^5}$

\item (Парадокс дней рождений) Найти вероятность того, что в классе из 23 человек, не менее двух учеников родились в один день.
\textbf{Решение:}
Пойдем от обратного и найдем вероятность того, что все в классе родились в разные дни. Рассмотрели одного ученика и закрепили его день рождения. Вероятность того, что 2ой ученик родился в другой день: $(1-\frac{1}{365})$. Вероятность того, что 3ий ученик родился не с первым и не со вторым: $(1-\frac{2}{365})$. Обобщая для n-ого: $(1-\frac{n}{365})$. Все эти события происходят одновременно, поэтому искомая вероятность: 
$$P = \prod_{n=1}^{22} (1-\frac{n}{365}) = \frac{365!}{365^n(365-n)!}$$ 
При $n=23$, $P=0.5$. 
\end{enumerate}
\end{document}

